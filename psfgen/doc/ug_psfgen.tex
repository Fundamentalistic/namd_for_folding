\section{Creating PSF Structure Files}
\label{section:psfgen}

The \PSFGEN\ structure building tool consists of a portable library
of structure and file manipulation routines with a Tcl interface.
Current capabilities include
\begin{itemize}
\item reading CHARMM topology files
\item reading psf files in X-PLOR/NAMD format
\item extracting sequence data from single segment PDB files
\item generating a full molecular structure from sequence data
\item applying patches to modify or link different segments
\item writing NAMD and VMD compatible PSF structure files
\item extracting coordinate data from PDB files
\item constructing (guessing) missing atomic coordinates
\item deleting selected atoms from the structure
\item writing NAMD and VMD compatible PDB coordinate files
\end{itemize}

We are currently refining the interface of \PSFGEN\ and adding
features to create a complete molecular building solution.
We welcome your feedback on this new tool.

\section{Ordinary Usage}

\PSFGEN\ is currently distributed in two forms.  One form is as a standalone
program implemented as a Tcl interpreter which reads
commands from standard output.  You may use loops, variables, etc. as
you would in a VMD or NAMD script.  You may use psfgen interactively,
but we expect it to be run most often with a script file redirected to
standard input.  The second form is as a Tcl package which can be imported
into any Tcl application, including \VMD.  All the commands available to the
standalone version of psfgen are available to the Tcl package; using \PSFGEN\
within \VMD\ lets you harness \VMD's powerful atom selection capability, as well
as instantly view the result of your structure building scripts.  Examples
of using \PSFGEN\ both with and without \VMD\ are provided in this document.

Generating PSF and PDB files for use with \NAMD\ will typically consist of
the following steps:

\begin{enumerate}
\item Preparing separate PDB files containing individual segments of
protein, solvent, etc. before running \PSFGEN.
\item Reading in the appropriate topology definition files and aliasing
residue and atom names found in the PDB file to those found in the topology
files.  This will generally include selecting a default protonation state
for histidine residues.
\item Generating the default structure using segment and pdb commands.
\item Applying additional patches to the structure.
\item Reading coordinates from the PDB files.
\item Deleting unwanted atoms, such as overlapping water molecules.
\item Guessing missing coordinates of hydrogens and other atoms.
\item Writing PSF and PDB files for use in \NAMD.
\end{enumerate}

\section{List of Commands}

\begin{itemize}

\item \COMMAND{\KEY{topology} \ARG{file name}}
{Read in molecular topology definitions from file.}
{\ARGDEF{file name}{CHARMM format topology file.}}
{Beginning of script, before segment.  May call multiple times.}

\item \COMMAND{\KEY{alias residue} \ARG{alternate name} \ARG{real name}}
{Provide translations from residues found in PDB files to proper
residue names read in from topology definition files.  Proper names
from topology files will be used in generated PSF and PDB files.}
{\ARGDEF{alternate name}{Residue name found in PDB file.}\\
\ARGDEF{real name}{Residue name found in topology file.}}
{Before reading sequence with pdb.  May call multiple times.}

\item \COMMAND{\KEY{segment} \ARG{segment ID} \{ \ARG{commands} \}}
{Build a segment of the molecule.  A segment is typically a single
chain of protein or DNA, with default patches applied to the termini.
Segments may also contain pure solvent or lipid.}
{\ARGDEF{segment ID}{Unique name for segment, 1--4 characters.}\\
\ARGDEF{commands}{Sequence of commands in Tcl syntax to build the primary
structure of the segment, including auto, first, last, residue, pdb, etc.}}
{After topology definitions and residue aliases.  May call multiple times.
Structure information is generated at the end of every segment command.}

\item \COMMAND{\KEY{auto} \OKEY{angles} \OKEY{dihedrals} \OKEY{none}}
{Override default settings from topology file for automatic generation of
angles and dihedrals for the current segment.}
{\KEYDEF{angles}{Enable generation of angles from bonds.}\\
\KEYDEF{dihedrals}{Enable generation of dihedrals from angles.}\\
\KEYDEF{none}{Disable generation of angles and dihedrals.}}
{Anywhere within segment, does not affect later segments.}

\item \COMMAND{\KEY{first} \ARG{patch name}}
{Override default patch applied to first residue in segment.
Default is read from topology file and may be residue-specific.}
{\ARGDEF{patch name}{Single-target patch residue name or \KEY{none}.}}
{Anywhere within segment, does not affect later segments.}

\item \COMMAND{\KEY{last} \ARG{patch name}}
{Override default patch applied to last residue in segment.
Default is read from topology file and may be residue-specific.}
{\ARGDEF{patch name}{Single-target patch residue name or \KEY{none}.}}
{Anywhere within segment, does not affect later segments.}

\item \COMMAND{\KEY{residue} \ARG{resid} \ARG{resname}}
{Add a single residue to the end of the current segment.}
{\ARGDEF{resid}{Unique name for residue, 1--5 characters, usually numeric.}
\ARGDEF{resname}{Residue type name from topology file.}}
{Anywhere within segment.}

\item \COMMAND{\KEY{pdb} \ARG{file name}}
{Extract sequence information from PDB file when building segment.
Residue IDs will be preserved, residue names must match entries in
the topology file or should be aliased before pdb is called.}
{\ARGDEF{file name}{PDB file containing known or aliased residues.}}
{Anywhere within segment.}

\item \COMMAND{\KEY{mutate} \ARG{resid} \ARG{resname}}
{Change the type of a single residue in the current segment.}
{\ARGDEF{resid}{Unique name for residue, 1--5 characters, usually numeric.}
\ARGDEF{resname}{New residue type name from topology file.}}
{Within segment, after target residue has been created.}

\item \COMMAND{\KEY{patch} \ARG{patch residue name} \ARG{segid:resid} \OARG{...}}
{Apply a patch to one or more residues.  Patches make small modifications to
the structure of residues such as converting one to a terminus, changing the
protonation state, or creating disulphide bonds between a pair of residues.}
{\ARGDEF{patch residue name}{Name of patch residue from topology definition file.}\\
\ARGDEF{segid:resid}{List of segment and residue pairs to which patch should be applied.}}
{After one or more segments have been built.}

\item \COMMAND{\KEY{multiply} \ARG{factor} \ARG{segid\OARG{{\em :resid}\OARG{:atomname}}} \OARG{...}}
{Create multiple images of a set of atoms for use in locally enhanced sampling.  The beta column of the output pdb file is set to 1...\ARG{factor} for each image.  Multiple copies of bonds, angles, etc. are created.  Atom, residue or segment names are not altered; images are distinguished only by beta value.  This is not a normal molecular structure and may confuse other tools.}
{\ARGDEF{factor}{}\\
\ARGDEF{segid:resid:atomname}{segment, residue, or atom to be multiplied.  If :resid is omitted the entire segment is multiplied; if :atomname is omitted the entire residue is multiplied.  May be repeated as many times as necessary to include all atoms.}}
{After one or more segments have been built, all patches applied, and coordinates guessed.  The effects of this command may confuse other commands.}

\item \COMMAND{\KEY{delatom} \ARG {segid} \OARG{resid} \OARG{atom name}}
{Delete one or more atoms.  If only {\tt segid} is specified, all atoms from
that segment will be removed from the structure.  If both {\tt segid} and
{\tt resid} are specified, all atoms from just that residue will be removed.
If {\tt segid}, {\tt resid}, and {\tt atom name} are all specified, just a
single atom will be removed.}
{\ARGDEF{segid}{Name of segment.}\\
\ARGDEF{resid}{Name of residue (optional).}\\
\ARGDEF{atom name}{Name of atom (optional).}}
{After all segments have been built and patched.}

\item \COMMAND{\KEY{resetpsf}}
{Delete all segments in from the structure.  The topology definitions and 
aliases are left intact.}
{}
{After one or more segments have been built.}

\item \COMMAND{\KEY{writepsf} \OKEY{charmm} \OKEY{x-plor} \ARG{file name}}
{Write out structure information as PSF file.}
{ \KEYDEF{charmm}{Use CHARMM format (numbers for atom types).}\\
\KEYDEF{x-plor}{Use X-PLOR format (names for atom types), the default format required by NAMD.}\\
\ARGDEF{file name}{PSF file to be generated.}}
{After all segments have been built and patched.}

\item \COMMAND{\KEY{readpsf} \ARG{file name}}
{Read in structure information from PSF file and adds it to the structure.
It is an error if any segments in the PSF file already exist.}
{\ARGDEF{file name}{PSF file in X-PLOR format (names for atom types).}}
{Anywhere but within segment.}

\item \COMMAND{\KEY{alias atom} \ARG{residue name} \ARG{alternate name} \ARG{real name}}
{Provide translations from atom names found in PDB files to proper
atom names read in from topology definition files.  Proper names
from topology files will be used in generated PSF and PDB files.}
{\ARGDEF{residue name}{Proper or aliased residue name.}\\
\ARGDEF{alternate name}{Atom name found in PDB file.}\\
\ARGDEF{real name}{Atom name found in topology file.}}
{Before reading coordinates with coordpdb.  May call multiple times.}

\item \COMMAND{\KEY{coord} \ARG{segid} \ARG{resid} \ARG{atomname} \ARG{\{ x y z \}}}
{Set coordinates for a single atom.}
{\ARGDEF{segid}{Segment ID of target atom.}\\
\ARGDEF{resid}{Residue ID of target atom.}\\
\ARGDEF{atomname}{Name of target atom.}\\
\ARGDEF{\{ x y z \}}{Coordinates to be assigned.}}
{After structure has been generated.}

\item \COMMAND{\KEY{coordpdb} \ARG{file name} \OARG{segid}}
{Read coordinates from PDB file, matching segment, residue and atom names.}
{\ARGDEF{file name}{PDB file containing known or aliased residues and atoms.}\\
\ARGDEF{segid}{If specified override segment IDs in PDB file.}}
{After segment has been generated and atom aliases defined.}

\item \COMMAND{\KEY{guesscoord}}
{Guesses coordinates of atoms for which they were not explicitly set.
Calculation is based on internal coordinate hints contained in toplogy
definition files.  When these are insufficient, wild guesses are attempted
based on bond lengths of 1 \AA\ and angles of 109$^\circ$.}
{None.}
{After stucture has been generated and known coordinates read in.}

\item \COMMAND{\KEY{writepdb} \ARG{file name}}
{Writes PDB file containing coordinates.  Atoms order is identical to
PSF file generated by writepsf (unless structure has been changed).
The O field is set to 1 for atoms with known coordinates, 0 for atoms
with guessed coordinates, and -1 for atoms with no coordinate data
available (coordinates are set to 0 for these atoms).}
{\ARGDEF{file name}{PDB file to be written.}}
{After structure and coordinates are complete.}

\end{itemize}

\section{BPTI Example}

To actually run this demo requires
\begin{itemize}
\item the program \verb#psfgen# from any \NAMD\ distribution,
\item the CHARMM topology and parameter files \verb#top_all22_prot.inp# and
\verb#par_all22_prot.inp# from 
{\tt https://rxsecure.umaryland.edu/research/amackere/research.html}, and
\item the BPTI PDB file \verb#6PTI.pdb# available from the Protein Data Bank at
{\tt http://www.pdb.org/} by searching for \verb#6PTI# and downloading
the complete structure file in PDB format.
\end{itemize}

In this demo, we create the files \verb#bpti.psf# and \verb#bpti.pdb# in the output directory
which can then be used for a simple NAMD simulation.

Create the working directory.  Nothing outside of the directory \verb#output# is modified.

\begin{verbatim}
mkdir output
\end{verbatim}

\paragraph*{Splitting input PDB file into segments.}

6PTI.pdb is the original file from the Protein Data Bank.  It contains
a single chain of protein and some PO4 and H2O HETATM records.  Since
each segment must have a separate input file, we remove all non-protein
atom records using grep.  If there were multiple chains we would have
to split the file by hand.
\begin{verbatim}
grep -v '^HETATM' 6PTI.pdb > output/6PTI_protein.pdb
\end{verbatim}

Create a second file containing only waters.
\begin{verbatim}
grep 'HOH' 6PTI.pdb > output/6PTI_water.pdb
\end{verbatim}


Run the psfgen program, taking everything until "ENDMOL" as input.
You may run psfgen interactively as well.  Since psfgen is built on
a Tcl interpreter, you may use loops, variables, etc., but you must
use \verb#$$# for variables when inside a shell script.  If you
want, run psfgen and enter the following commands manually.

\begin{verbatim}
psfgen << ENDMOL
\end{verbatim}

\paragraph*{Reading topology file.}

Read in the topology definitions for the residues we will create.
This must match the parameter file used for the simulation as well.
Multiple topology files may be read in since psfgen and NAMD use atom
type names rather than numbers in psf files.

\begin{verbatim}
topology toppar/top_all22_prot.inp
\end{verbatim}

\paragraph*{Building segment BPTI.}


Actually build a segment, calling it BPTI and reading the sequence
of residues from the stripped pdb file created above.  In addition to
the pdb command, we could specify residues explicitly.  Both angles
and dihedrals are generated automatically unless "auto none" is added
(which is required to build residues of water).  The commands "first"
and "last" may be used to change the default patches for the ends of
the chain.  The structure is built when the closing \} is encountered,
and some errors regarding the first and last residue are normal.

\begin{verbatim}
segment BPTI {
 pdb output/6PTI_protein.pdb
}
\end{verbatim}

\paragraph*{Adding patches.}


Some patch residues (those not used to begin or end a chain) are
applied after the segment is built.  These contain all angle and
dihedral terms explicitly since they were already generated.  In this
case we apply the patch for a disulfide link three separate times.

\begin{verbatim}
patch DISU BPTI:5 BPTI:55
patch DISU BPTI:14 BPTI:38
patch DISU BPTI:30 BPTI:51
\end{verbatim}

\paragraph*{Reading coordinates from pdb file.}


The same file used to generate the sequence is now read to extract
coordinates.  In the residue ILE, the atom CD is called CD1 in the
pdb file, so we use "alias atom" to define the correct name.  Segment
names in the pdb file are ignored so we specify that the coordinates
should be applied to the segment BPTI.

\begin{verbatim}
alias atom ILE CD1 CD
coordpdb output/6PTI_protein.pdb BPTI
\end{verbatim}

\paragraph*{Adding a segment of water.}


Build a segment for the crystal waters.  The residue type for water
depends on the model, so here we alias HOH to TIP3.  Because CHARMM
uses an additional H-H bond we must disable generation of angles and
dihedrals for segments containing water.  Then read the pdb file.

\begin{verbatim}
alias residue HOH TIP3
segment SOLV {
 auto none
 pdb output/6PTI_water.pdb
}
\end{verbatim}

\paragraph*{Reading water coordinates.}


Alias the atom type for water oxygen as well and read coordinates from
the file to the segment SOLV.  Hydrogen doesn't show up in crystal
structures so it is missing from this pdb file.

\begin{verbatim}
alias atom HOH O OH2
coordpdb output/6PTI_water.pdb SOLV
\end{verbatim}


\paragraph*{Writing psf structure file.}


Now that all of the atoms and bonds have been created, we can write
out the psf structure file for the system.

\begin{verbatim}
writepsf output/bpti.psf
\end{verbatim}

\paragraph*{Guessing missing coordinates.}


The tolopogy file contains default internal coordinates which can be
used to guess the locations of many atoms, hydrogens in particular.
In the output pdb file, the occupancy field of guessed atoms will be
set to 0, atoms which are known are set to 1, and atoms which could
not be guessed are set to -1.  Some atoms are "poorly guessed" if
needed bond lengths and angles were missing from the topology file.
Similarly, waters with missing hydrogen coordinates are given a
default orientation.

\begin{verbatim}
guesscoord
\end{verbatim}

\paragraph*{Writing pdb coordinate file.}


This creates the matching coordinate pdb file.  The psf and pdb files
are a matched set with identical atom ordering as needed by NAMD.

\begin{verbatim}
writepdb output/bpti.pdb
\end{verbatim}

\begin{verbatim}
ENDMOL
\end{verbatim}


\paragraph*{Using generated files in NAMD.}

The files bpti.pdb and bpti.psf can now be used with \NAMD, but the
initial coordinates require minimization first.
The following is an example \NAMD\ configuration file for the BPTI example.

%\newpage
\begin{verbatim}
# NAMD configuration file for BPTI

# molecular system
structure	output/bpti.psf

# force field
paratypecharmm	on
parameters	toppar/par_all22_prot.inp
exclude		scaled1-4
1-4scaling	1.0

# approximations
switching	on
switchdist	8
cutoff		12
pairlistdist	13.5
margin		0
stepspercycle	20

#integrator
timestep 1.0

#output
outputenergies	10
outputtiming	100
binaryoutput	no

# molecular system
coordinates	output/bpti.pdb

#output
outputname	output/bpti
dcdfreq		1000

#protocol
temperature	0
reassignFreq	1000
reassignTemp	25
reassignIncr	25
reassignHold	300

#script

minimize 1000

run 20000
\end{verbatim}

\section{Building solvent around a protein}
The following script illustrates how \PSFGEN\ and \VMD\ can be used together
to add water around a protein structure.  It assumes you already have a 
psf and pdb file for your protein, as well as a box of water which is 
large enough to contain the protein. For more information on how atomselections
can be used within \VMD\ scripts, see the \VMD\ User's Guide.

\begin{verbatim}
proc addwater { psffile pdbfile watpsf watpdb } {
	# Create psf/pdb files that contain both our protein as well as
	# a box of equilibrated water.  The water box should be large enough
	# to easily contain our protein.
	resetpsf
	readpsf $psffile
	readpsf $watpsf
	coordpdb $pdbfile
	coordpdb $watpdb

	# Load the combined structure into VMD   
	writepsf combine.psf
	writepdb combine.pdb
	mol load psf combine.psf pdb combine.pdb

	# Assume that the segid of the water in watpsf is QQQ
	# We want to delete waters outside of a box ten Angstroms
	# bigger than the extent of the protein. 
	set protein [atomselect top "not segid QQQ"]
	set minmax [measure minmax $protein]
	foreach {min max} $minmax { break }
	foreach {xmin ymin zmin} $min { break }
	foreach {xmax ymax zmax} $max { break }
    set xmin [expr $xmin - 10]
    set ymin [expr $ymin - 10]
    set zmin [expr $zmin - 10]
    set xmax [expr $xmax + 10]
    set ymax [expr $ymax + 10]
    set zmax [expr $zmax + 10]

	# Center the water on the protein.  Also update the coordinates held
	# by psfgen.
	set wat [atomselect top "segid QQQ"]
	$wat moveby [vecsub [measure center $protein] [measure center $wat]]
	foreach atom [$wat get {segid resid name x y z}] {
		foreach {segid resid name x y z} $atom { break }
		coord $segid $resid $name [list $x $y $z]
	}

	# Select waters that we don't want in the final structure.
	set outsidebox [atomselect top "segid QQQ and (x <= $xmin or y <= $ymin \
		or z <= $zmin or x >= $xmax or y >= $ymax or z >= $xmax)"]
	set overlap [atomselect top "segid QQQ and within 2.4 of (not segid QQQ)"]

	# Get a list of all the residues that are in the two selections, and delete
	# those residues from the structure.
	set reslist [concat [$outsidebox get resid] [$overlap get resid]]
	set reslist [lsort -unique -integer $reslist]

	foreach resid $reslist {
		delatom QQQ $resid
	}

	# That should do it - write out the new psf and pdb file. 
	writepsf solvate.psf 
	writepdb solvate.pdb

	# Delete the combined water/protein molecule and load the system that
	# has excess water removed.
	mol delete top
	mol load psf solvate.psf pdb solvate.pdb

	# Return the size of the water box
	return [list [list $xmin $ymin $zmin] [list $xmax $ymax $zmax]]
}
\end{verbatim}

