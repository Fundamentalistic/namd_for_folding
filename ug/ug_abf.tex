

%%%%%%%%%%%%%%%%%%%%%%%%%%%%%%%%%%%%%%%%%%%%%%%%%%%%%%%%%%%%%%%%%%%%%%%%%%%%%%%%%
%%% TITLE
%%%%%%%%%%%%%%%%%%%%%%%%%%%%%%%%%%%%%%%%%%%%%%%%%%%%%%%%%%%%%%%%%%%%%%%%%%%%%%%%%


% \title{The adaptive biasing force method user's guide}

This feature has been contributed to \NAMD\ by the following authors:

\begin{quote}
   J\'er\^ome H\'enin and Christophe Chipot           \\[0.4cm]
   {\it Equipe de dynamique des assemblages membranaires, }\\
   {\it Institut nanc\'eien de chimie mol\'eculaire,      }\\
   {\it UMR CNRS/UHP 7565,                                }\\
   {\it Universit\'e Henri Poincar\'e,                    }\\
   {\it BP 239,                                           }\\
   {\it 54506 Vand\oe uvre--l\`es--Nancy cedex, France    }
\end{quote}

\copyright~2005, {\sc Centre National de la Recherche Scientifique}


%%%%%%%%%%%%%%%%%%%%%%%%%%%%%%%%%%%%%%%%%%%%%%%%%%%%%%%%%%%%%%%%%%%%%%%%%%%%%%%%%
%%% TEXT
%%%%%%%%%%%%%%%%%%%%%%%%%%%%%%%%%%%%%%%%%%%%%%%%%%%%%%%%%%%%%%%%%%%%%%%%%%%%%%%%%



\subsubsection{Introduction and theoretical background}


Strictly speaking, a potential of mean force (PMF) is the
reversible work supplied to the system to bring two solvated
particles, or ensembles of particles, from an infinite separation
to some contact distance:~\cite{chan_87_1}


\begin{equation}
\label{pmf}
w(r) = -\frac{1}{\beta} \ln g(r)
\end{equation}


Here, $g(r)$ is the pair correlation function of the
two particles, or ensembles thereof.
The vocabulary ``PMF'' has, however, been extended
to a wide range of reaction coordinates that go far
beyond simple interatomic or intermolecular distances.
In this perspective, generalization of equation~({\ref{pmf}})
is not straightforward. This explains why it may be
desirable to turn to a definition suitable for any
type of reaction coordinate, $\xi$:


\begin{equation}
\label{free}
{A(\xi) = -\frac{1}{\beta} \ln \pxi + A_0}
\end{equation}


$A(\xi)$ is the free energy of the state
defined by a particular value of $\xi$, which
corresponds to an iso--$\xi$ hypersurface in
phase space. $A_0$ is a constant and $\pxi$
is the probability density to find the
chemical system of interest at $\xi$.


The connection between
the derivative of the free energy with respect to
the reaction coordinate, $\dd A(\xi) / \dd\xi$,
and the forces exerted along the latter
may be written as:~\cite{spri_98_1, deno_00_1}


\begin{equation}
\label{derivative}
\frac{\dd A(\xi)}{\dd\xi} = \left\langle \frac{\dr\pot}{\dr\xi}
                     - \frac{1}{\beta}
                       \frac{\dr \ln |J|}{\dr\xi} \right\rangle_\xi
                     = -\langle F_\xi \rangle_\xi
\end{equation}


where $|J|$ is the determinant of the Jacobian for the
transformation from generalized to Cartesian coordinates. The
first term of the ensemble average corresponds to the Cartesian
forces exerted on the system, derived from the potential energy
function, $\pot$. The second contribution is a geometric
correction arising from the change in the metric of the phase
space due to the use of generalized coordinates.
It is worth noting, that, contrary to its instantaneous
component, $F_\xi$, only the average force,
$\langle F_\xi \rangle_\xi$, is physically meaningful.


In the framework of the average biasing force (ABF)
approach,~\cite{darv_01_1, rodr_04_1}
$F_\xi$ is accumulated in small windows
or bins of finite size, $\delta \xi$,
thereby providing an estimate of the derivative
$\dd A(\xi) / \dd\xi$ defined in equation~({\ref{derivative}}).
The force applied along the reaction coordinate,
$\xi$, to overcome free energy barriers is defined
by:


\begin{equation}
\label{abf}
{\bf F}^{\rm ABF} = \gradx \widetilde A
                  = - \langle F_\xi \rangle_\xi \ \gradx \xi
\end{equation}


where $\widetilde A$ denotes the current estimate of the
free energy and $\langle F_\xi \rangle_\xi$, the current
average of $F_\xi$.


As sampling of the phase space proceeds, the estimate
$\gradx \widetilde A$ is progressively refined. The biasing
force, ${\bf F}^{\rm ABF}$, introduced in the equations of
motion guarantees that in the bin centered about $\xi$,
the force acting along the reaction coordinate averages
to zero over time. Evolution of the system along $\xi$
is, therefore, governed mainly by its self--diffusion
properties.


A particular feature of the
instantaneous force, $F_\xi$,
is its tendency to fluctuate
significantly.
As a result, in the beginning of an ABF simulation,
the accumulated average in each bin will generally
take large, inaccurate
values. Under these circumstances,
applying the biasing force along $\xi$ according
to equation~({\ref{abf}}) may severely perturb
the dynamics of the system, thereby
biasing artificially the accrued average, and,
thus, impede convergence.
To avoid such undesirable effects,
no biasing force is applied in a bin centered about $\xi$
until a reasonable number of force samples
has been collected. When the user--defined minimum number
of samples is reached, the biasing force
is introduced progressively in the form
of a linear ramp.
For optimal efficiency, this minimal number of samples
should be adjusted on a system--dependent basis.


In addition, to alleviate the deleterious
effects caused by abrupt variations of the force,
the corresponding fluctuations are smoothed out,
using a weighted running average over a preset number
of adjacent bins, in lieu of the average of the
current bin itself. It is, however, crucial to ascertain
that the free energy profile varies regularly
in the $\xi$--interval, over which the
average is performed.

To obtain an adequate sampling in reasonable
simulation times, it is recommended to split long reaction
pathways into consecutive ranges of $\xi$. In contrast with
probability--based methods, ABF does
{\itshape not} require that these windows overlap by
virtue of the continuity of the force across the
the reaction pathway.


A more comprehensive discussion of the theoretical basis of
the method and its implementation
in \namd\ can be found in~\cite{heni_04_1}.



\subsubsection{Using the \namd\ implementation of the adaptive biasing force method}


The ABF method has been implemented as a suite of \tcl \
routines that can be invoked from the main configuration file used
to run molecular dynamics simulations with \namd.

The routines can be invoked by including the following
command in the configuration file:

\begin{verbatim}
source <path to>/lib/abf/abf.tcl
\end{verbatim}

where \texttt{<path to>/lib/abf/abf.tcl} is the complete path to
the main ABF file. The other \tcl\ files of the ABF
package must be located in the same directory.

A second option for loading the ABF module is to source the
\texttt{lib/init.tcl} file distributed with NAMD, and then use the
Tcl package facility.  The file may be sourced in the config file:

\begin{verbatim}
source <path to>/lib/init.tcl
package require abf
\end{verbatim}

Or, to make the config file portable, the init file may be the
first config file passed to NAMD:

\begin{verbatim}
namd2 <path to>/lib/init.tcl myconfig.namd
\end{verbatim}

and then the config file need only contain:

\begin{verbatim}
package require abf
\end{verbatim}

Note that the ABF code makes use of the {\tt TclForces} and
{\tt TclForcesScript} commands.
As a result, \namd\ configuration files
should not call the latter directly when running ABF.



\subsubsection{Parameters for ABF simulations}


The following parameters have been defined to control ABF \
calculations. They may be set using the folowing syntax:

{\tt abf <keyword> <value>}
\index{abf command}

where {\tt <value>} may be a number, a word or a \tcl\ list.
Keywords are not case--sensitive.


\begin{itemize}
\setlength{\itemsep}{0.4cm}

\item
\NAMDCONF{coordinate}{Type of reaction coordinate used in the ABF \
                         simulation}
{{\tt distance}, {\tt distance-com}, {\tt abscissa}, {\tt zCoord}, 
{\tt zCoord-1atom} or {\tt xyDistance}}
{As a function of the system of interest, a number of alternative
 reaction coordinates may be considered. The ABF code is modular:
 RC--specific code for each RC\ is contained in a separate \tcl\
 file.
 Additional RCs, tailored for a particular problem, may be added
 by creating a new file providing the required \tcl\ procedures.
 Existing files such as {\tt distance.tcl} are thoroughly
 commented, and should provide a good basis for the coding of
 additional coordinates, together with~\cite{heni_04_1}.
 \begin{enumerate}
 \renewcommand{\itemsep}{0.4cm}
 \renewcommand{\labelenumi}{(\arabic{enumi})}
 \item
 {\tt distance} \hfill
 \begin{minipage}[t]{10.0cm}
                    corresponds to the distance separating two selected
                    atoms:
                    \newline
                    {\tt abf1}: index of the
                                first atom of the reaction coordinate;
                    \newline
                    {\tt abf2}: index of the
                                second atom of the reaction coordinate.
 \end{minipage}
 \item
 {\tt distance-com} \hfill
 \begin{minipage}[t]{10.0cm}
                    corresponds to the distance separating the center of
                    mass of two sets of atoms, \eg the distance between the
                    centroid of two benzene molecules:
                    \newline
                    {\tt abf1}: list of indices of atoms participating
                                to the first center of mass;
                    \newline
                    {\tt abf2}: list of indices of atoms participating
                                to the second center of mass.
 \end{minipage}
 \item
 {\tt abscissa} \hfill
 \begin{minipage}[t]{10.0cm}
                    corresponds to the distance between the centers of mass
            of two sets of atoms along a given direction:
                    \newline
                    {\tt direction}: a vector (\tcl\ list of three real numbers)
                defining the direction of interest
                    \newline
                    {\tt abf1}: list of indices of atoms participating
                                to the first center of mass;
                    \newline
                    {\tt abf2}: list of indices of atoms participating
                                to the second center of mass.
 \end{minipage}
 \item
 {\tt zCoord} \hfill
 \begin{minipage}[t]{10.0cm}
                    corresponds to the distance separating two groups of atoms
                    along the $z$-direction of
                    Cartesian space. This reaction coordinate is
                    useful for the estimation of transfer free energies
                    between two distinct media:
                    \newline
                    {\tt abf1}: list of indices of reference atoms
                    \newline
                    {\tt abf2}: list of indices of atoms of interest ---
                                \eg a solute
 \end{minipage}
 \item
 {\tt zCoord-1atom} \hfill
 \begin{minipage}[t]{10.0cm}
                    is similar to {\tt zCoord}, but using a single
                    atom of interest
                    \newline
                    {\tt abf1}: list of indices of reference atoms
                    \newline
                    {\tt abf2}: index of an atom of interest
 \end{minipage}
 \item
 {\tt xyDistance} \hfill
 \begin{minipage}[t]{10.0cm}
                    is the distance between the centers of mass of
		    two atom groups, projected on
		    the $(x, y)$ plane:
                    \newline
                    {\tt abf1}: list of indices of atoms participating
                                to the first center of mass;
                    \newline
                    {\tt abf2}: list of indices of atoms participating
                                to the second center of mass.
 \end{minipage}
 \end{enumerate}
 }


\item
\NAMDCONF{xiMin}{Lower bound of the reaction coordinate}
{decimal number, in \AA}
{Lower limit of the reaction coordinate, $\xi$, below which no
 sampling is performed.}


\item
\NAMDCONF{xiMax}{Upper bound of the reaction coordinate}
{decimal number, in \AA}
{Upper limit of the reaction coordinate, $\xi$, beyond which no
 sampling is performed.}


\item
\NAMDCONF{dxi}{Width of the bins in which the forces are accumulated}
{decimal number, in \AA}
{Width, $\delta\xi$, of the bins in which the instantaneous components
 of the force, $F_\xi$, are collected. The choice of this
 variable is dictated by the nature of the system and how rapidly
 the free energy changes as a function of $\xi$.}


\item
\NAMDCONFWDEF{forceConst}{Force constant applied at the borders of the
                          reaction coordinate}
{positive decimal number, in kcal/mol/\AA$^2$}
{10.0}
{If this force constant is nonzero, a harmonic bias is enforced at
 the borders of the reaction coordinate,
 \ie at both {\tt xiMin} and {\tt xiMax}, to guarantee that sampling
 will be confined between these two values.}


\item
\NAMDCONFWDEF{dSmooth}{Length over which force data is averaged when computing
                       the ABF}
{positive decimal number, in \AA}
{0.0}
{To avoid abrupt variations in the average force and in the free energy,
 fluctuations are smoothed out by means of a weighted running average over
 adjacent bins on each side of $\xi$. When the free energy derivative varies
 slowly, smoothing can be performed across several contiguous bins. Great attention
 should be paid, however, when the free energy varies sharply with $\xi$.}


\item \NAMDCONFWDEF{fullSamples}{Number of samples in a bin prior
              to application of the ABF}
{positive integer}
{200}
{To avoid nonequilibrium effects in the dynamics of the system, due to large
 fluctuations of the force exerted along the reaction coordinate, $\xi$, it
 is recommended to apply the biasing force only after a reasonable estimate
 of the latter has been obtained.}


\item
\NAMDCONFWDEF{outFile}{Output file of an ABF calculation}
{\unix \ filename}
{\texttt{abf.dat}}
{Output file containing, for each value of the reaction
 coordinate, $\xi$, the free energy, $A(\xi)$, the average
 force, $\left\langle F_\xi \right\rangle$, and the number
 of samples accrued in the bin.}


\item
\NAMDCONFWDEF{historyFile}{History of the force data over time}
{\unix \ filename}
{\texttt{none}}
{This file contains a history of the information stored in
\texttt{outFile}, and is written every tenth update of \texttt{outFile}.
Data for different timesteps are separated by \& signs for easier
visualization using the Grace plotting software.
This is useful for assessing the convergence of the free energy
profile.}


\item
\NAMDCONFWDEF{inFiles}{Input files for an ABF calculation}
{\tcl\ list of \unix\ filenames}
{empty list}
{Input files containing the same data as in {\tt outFileName}, and
 that may, therefore, be used to restart an ABF simulation. The
 new free energies and forces will then be written out based on the
 original information supplied by these files. This command may also
 be used to combine data obtained from separate runs.}


\item
\NAMDCONFWDEF{outputFreq}{Frequency at which {\tt outFileName} is
                          updated}
{positive integer}
{5000}
{The free energy, $A(\xi)$, the average
 force, $\left\langle F_\xi \right\rangle$, and the number
 of samples accrued in the bin will be written to the ABF \
 output file every \texttt{ABFoutFreq} timesteps.}


\item
\NAMDCONFWDEF{writeXiFreq}{Frequency at which the time series of $\xi$ is
                           written}
{positive integer}
{0}
{If this parameter is nonzero, the instantaneous value of the
 reaction coordinate, $\xi$, is written in the \namd \ standard
 output every \texttt{writeXiFreq} time steps.}


\item
\NAMDCONFWDEF{distFile}{Output file containing force distributions}
{\unix \ filename}
{\none}
{Output file containing a distribution of the instantaneous
 components of the force, $F_\xi$, for every bin comprised
 between {\tt xiMin} and {\tt xiMax}. This is useful for
 performing error analysis for the resulting free energy profile}


\item
\NAMDCONFWDEF{fMax}{Half--width of the force histograms}
{positive decimal number, in kcal/mol/\AA}
{60.0}
{When force distributions are written in {\tt distFile},
 the histogram collects $F_\xi$ values ranging
 from $-${\tt fMax} to $+${\tt fMax}.}


\item
\NAMDCONFWDEF{moveBoundary}{Number of samples beyond which {\tt xiMin} and {\tt xiMax}
                            are updated}
{positive integer}
{0}
{Slow relaxation in the degrees of freedom orthogonal to the reaction
 coordinate, $\xi$, results in a non--uniform sampling along the latter.
 To force exploration of $\xi$ in quasi non--ergodic situations, the
 boundaries of the reaction pathway may be modified dynamically when a
 preset minimum number of samples is attained. As a result, the
 interval between {\tt xiMin} and {\tt xiMax} is progressively narrowed
 down as this threshold is being reached for all values of $\xi$.
 It should be clearly understood that uniformity of the sampling
 is artificial and is only used to force diffusion along $\xi$. Here,
 uniform sampling does not guarantee the proper convergence of the
 simulation. }


\item
\NAMDCONFWDEF{restraintList}{Apply external restraints}
{list of formatted entries}
{empty list}
{For a detailed description of
 this feature, see next subsection.}


\item
\NAMDCONFWDEF{applyBias}{Apply a biasing force in the simulation~?}
{\yes \ or \no}
{\yes}
{By default, a biasing force is applied along the reaction coordinate,
 $\xi$, in ABF calculations. It may be, however, desirable to set
 this option to \no \ to monitor the evolution of the system along
 $\xi$ and collect the forces and the free energy, yet,
 without introducing any bias in the simulation.}


\item
\NAMDCONFWDEF{usMode}{Run umbrella sampling simulation~?}
{\yes \ or \no}
{\no}
{When setting this option, an ``umbrella sampling''~\cite{torr_77_2}
 calculation
 is performed, supplying a probability distribution
 for the reaction coordinate, $\xi$. As an initial
 guess of the biases required to overcome the free energy
 barriers, use can be made of a previous ABF simulation ---
 \cf {\tt inFiles}. In addition, specific restraints may
 be defined using the {\tt restraintList} feature described below.}

\end{itemize}



\subsubsection{Including restraints in ABF simulations}


In close connection with the possibility to run umbrella sampling simulations, the
ABF module of \namd \ also includes the capability to add sets of
restraints to confine the system in selected regions of configurational
space. Incorporation of harmonic restraints may be invoked using a syntax
similar in spirit to that adopted in the conformational free energy
module of \namd:


\begin{verbatim}
abf restraintList {
   angle1  {angle {A 1 CA} {G 1 N3} {G 1 N1}          40.0 30.0}
   dihe1   {dihe  {A 1 O1} {A 1 CA} {A 1 O2} {G 1 N3} 40.0  0.0}
   dihe2   {dihe  {A 1 CA} {A 1 O2} {G 1 N2} {G 1 N3} 40.0  0.0}
}
\end{verbatim}


The general syntax of an item of this list is:

\begin{verbatim}
name {type atom1 atom2 [atom3] [atom4] k reference}
\end{verbatim}


where {\tt name} could be any word used to refer to the restraint in the
\namd \ output. {\tt type} is the type of restraint, \ie a distance,
{\tt dist}, a valence angle, {\tt angle}, or a dihedral angle, {\tt dihe}.
Definition of the successive atoms follows the syntax of \namd \
conformation free energy calculations. {\tt k} is the force constant
of the harmonic restraint, the unit of which depends on
{\tt type}. {\tt reference} is the reference, target value of
the restrained coordinate.


Aside from {\tt dist}, {\tt angle} and {\tt dihe}, which correspond to
harmonic restraints, linear ramps have been added for distances, using
the specific keyword {\tt distLin}. In this particular case, generally
useful in umbrella sampling free energy calculations, the syntax is:


\begin{verbatim}
name {distLin atom1 atom2 F r1 r2}
\end{verbatim}


where {\tt F} is the force in kcal/mol/\AA. The restraint is
applied over the interval between {\tt r1} and {\tt r2}.

The {\tt harm} restraint applies, onto one single atom, a harmonic potential
of the form\\
\begin{displaymath}
V(x,y,z) = \frac{1}{2} k_x (x-x_0)^2
+ \frac{1}{2} k_y (y-y_0)^2 + \frac{1}{2} k_z (z-z_0)^2
\end{displaymath}

This way, atoms may be restrained to the vicinity of a plane, an axis, or
a point, depending on the number of nonzero force constants. The syntax is:
\begin{verbatim}
name {harm atom {kx ky kz} {x0 y0 z0} }
\end{verbatim}



\subsubsection{Important recommendations when running ABF simulations}


The formalism implemented in \namd \ had been originally devised in the
framework of unconstrained MD. Holonomic constraints may, however,
be introduced without interfering with the computation of the bias and, hence,
the PMF, granted that some precautions are taken.


Either of the following strategies may be adopted:
\begin{enumerate}
\renewcommand{\labelenumi}{(\arabic{enumi})}
\item Atoms involved in the computation of $F_\xi$ do not participate in
constrained degrees of freedom.
If, for instance, chemical bonds between hydrogen and heavy
atoms are frozen, $\xi$ could
be chosen as a distance between atoms that are not involved
in a constraint.


In some cases, not all atoms used for defining $\xi$ are involved in
the computation of the force component $F_\xi$. Specifically, reference
atoms \texttt{abf1} in the RC\ types \texttt{zCoord} and \texttt{zCoord-1atom}
are not taken into account in $F_\xi$. Therefore, constraints involving
those atoms have no effect on the ABF protocol.


\item The definition of $\xi$ involves atoms forming constrained bonds.
In this case, the effect of constraints on ABF can be eliminated
by using a group--based RC\ built on atom groups
containing both ``ends'' of all rigid bonds involved, \ie hydrogens
together with their mother atom.
\end{enumerate}


For example, if the distance from a methane molecule to the center
of mass of a water lamella is studied with the \texttt{zCoord}
RC\ by taking water oxygens as a reference (\texttt{abf1}) and
all five atoms of methane as the group of interest (\texttt{abf2}):
\begin{itemize}
\item Water molecules may be constrained because the reference
atoms are not used to compute the force component;
\item C---H bonds in methane may be constrained as well, because
the contributions of constraint forces on the carbon and hydrogens
to $F_\xi$ will cancel out.
\end{itemize}



\subsubsection{Example of input file for computing potentials of mean force}


In this example,
the system consists of a short,
ten-residue peptide, formed by {\sc l}-alanine amino acids.
Reversible folding/unfolding of
this peptide is carried out using as a reaction coordinate the carbon
atom of the first and the last carbonyl groups --- $\xi$, hence,
corresponds to a a simple interatomic distance, defined by the
keyword {\tt distance}. The reaction pathway is explored between
12 and 32~\AA, \ie, over a distance of 20~\AA, in a single window.
The variations of the free energy derivative are soft enough to warrant
the use of bins with a width of 0.2~\AA, in which forces are accrued.


\begin{verbatim}
source                ~/lib/abf/abf.tcl

abf coordinate        distance

abf abf1              4
abf abf2              99

abf dxi               0.2
abf xiMin             12.0
abf xiMax             32.0
abf outFile           deca-alanine.dat
abf fullSamples       500
abf inFiles           {}
abf distFile          deca-alanine.dist
abf dSmooth           0.4
\end{verbatim}


Here, the ABF is applied after 500 samples have been
collected, which, in vacuum, have proven to be sufficient
to get a reasonable estimate of $\left\langle F_\xi \right\rangle$.

