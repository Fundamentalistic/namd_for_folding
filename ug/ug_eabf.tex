

%%%%%%%%%%%%%%%%%%%%%%%%%%%%%%%%%%%%%%%%%%%%%%%%%%%%%%%%%%%%%%%%%%%%%%%%%%%%%%%%%
%%% TITLE
%%%%%%%%%%%%%%%%%%%%%%%%%%%%%%%%%%%%%%%%%%%%%%%%%%%%%%%%%%%%%%%%%%%%%%%%%%%%%%%%%


\subsection{Free energy estimator for the Extended Adaptive Biasing Force Method}

\noindent
This feature has been contributed to \NAMD\ by the following authors:

\begin{quote}
   Haohao Fu and Christophe Chipot                                                               \\[0.4cm]
   Laboratoire International Associ\'e 
   Centre National de la Recherche Scientifique et University of Illinois at Urbana--Champaign,   \\
   Unit\'e Mixte de Recherche No. 7565, Universit\'e de Lorraine,                                \\
   B.P. 70239, 54506 Vand\oe uvre-l�s-Nancy cedex, France
\end{quote}

\copyright~2016, {\sc Centre National de la Recherche Scientifique}


%%%%%%%%%%%%%%%%%%%%%%%%%%%%%%%%%%%%%%%%%%%%%%%%%%%%%%%%%%%%%%%%%%%%%%%%%%%%%%%%%
%%% TEXT
%%%%%%%%%%%%%%%%%%%%%%%%%%%%%%%%%%%%%%%%%%%%%%%%%%%%%%%%%%%%%%%%%%%%%%%%%%%%%%%%%


\subsubsection{Introduction and theoretical background}

In an adaptive biasing force (ABF)
simulation \cite{Darve2001,Henin2004},
the gradient of the 
free energy along the transition
coordinate of interest, $\xi$,
is defined as,

\begin{equation}
\frac{{\rm d}A}{{\rm d}\xi} =
\left\langle 
\frac{\partial U({\bf x})}{\partial \xi} - \frac{1}{\beta}
\frac{\partial \ln |J|}{\partial \xi}
\right\rangle_\xi
:= -\langle F_\xi \rangle_\xi
\end{equation}

\noindent
where $U({\bf x})$ is the potential energy function, $\beta = 1 / k_{\rm B}T$,
and $|J|$ is the determinant of the Jacobian for the generalized--to--Cartesian
coordinate transformation. To perform an ABF simulation, the following
three requirements should be met,

\begin{enumerate}
\renewcommand\labelenumi{(\roman{enumi})}
\renewcommand\theenumi\labelenumi
\item{}
Availability of the derivative of the Jacobian. 
Calculation of the biasing force, 
$-\langle F_\xi \rangle_\xi \nabla \xi$, requires the
analytical expression of the derivative of the Jacobian, 
$\partial \ln |J| / \partial \xi$,
which involves the second derivative of the coarse variable and
can sometimes take a mathematically complicated form.21 

\item{}
Independence of the coarse variables. In a multidimensional
ABF calculation, each coarse variable should be mutually
orthogonal. Similarly, when a coarse variable consists of a
combination of several components, these components should
also be mutually orthogonal.

\item{}
Orthogonality of the coarse variables and constraints or
restraints. If there are geometric restraints or holonomic
constraints in the assay, they should be fully decoupled from
the coarse variables. Should holonomic constraints be involved
in the model reaction coordinate, the incorrect free-energy
change stems from the contamination of the measured
thermodynamic force by the constraint forces. When geometric
restraints are coupled to the coarse variables, ABF is not
able to discriminate the origin of the forces at play, 
resulting in the application of an erroneous time-dependent 
bias.
\end{enumerate}

These stringent prerequisites can be circumvented
by turning to an extended Lagrangian formalism in an 
ABF simulation (See 10.2.4). 
This approach is referred to as extended ABF, 
or eABF method. 
In eABF approach, the gradient of the free energy 
is given by \cite{Zheng2012,Fu2016},
              
\begin{equation}
\label{eq:eABF}
\left(\frac{{\rm d}A}{{\rm d}\xi}\right)_{\xi'} =
\frac{\displaystyle \sum_{\Xi'} N(\xi', \Xi') 
\left[ 
\frac{(\xi' - \langle\xi_{\Xi'}\rangle)}{\beta \sigma_{\Xi'}^2} - K_{\xi} (\xi' - \Xi')
\right]}
{\displaystyle \sum_{\Xi'} N(\xi', \Xi')}
\end{equation}

\noindent
The transition coordinate, $\xi$, is a function of the positions 
of the real particles of the molecular assembly and is restrained 
through potential $1/2 \ K_{\xi} (\xi' - \xi_{\Xi'})^2$
to a one-dimensional fictitious particle, $\Xi$, moving dynamically. 
$N(\xi', \Xi')$ is the number of samples $\xi'$ collected 
from the $\Xi'$--restrained ensemble, which is assumed to be Gaussian.
At the present stage, equation \ref{eq:eABF} is implemented 
through the scripted Colvars interface (See 10.5.7) for one-- and 
two--dimensional free-energy calculations. 

%\medskip
\subsubsection{Implementation in NAMD}

To perform an eABF simulation, one needs to set 
{\tt scriptedColvarForces on} 
and source the eabf.tcl file found in the lib/eabf directory.
Here, an example of a configuration file is supplied
for an eABF simulation:

\bigskip

\noindent
\begin{minipage}[t]{6.5cm}
{\small 
\addtolength{\baselineskip}{0.1\baselineskip}
\begin{verbatim}
source                eabf.tcl
set eabf_inputname	   0	
set eabf_outputname	  output.eabf 
set eabf_temperature  300	
set eabf_outputfreq	  20000 
\end{verbatim}
}
\end{minipage} 
\hfill                
\begin{minipage}[t]{9.5cm}
Enables eABF;
\newline
Prefix for restart files. ``0'' is used for new runs;
\newline
Prefix for output files;
\newline
Temperature used in the calculation;
\newline
Frequency at which eABF data files are updated.
\end{minipage}                 

\medskip

\paragraph*{Multiple--replica eABF.}
The eABF algorithm can be associated with a multiple--walker strategy \cite{Minoukadeh2010,Comer2014c} (See 10.5.1). To run a multiple--replica 
eABF simulation, one can simply set {\tt shared on} in the Colvars config file to enable 
the multiple--walker algorithm. For example,

\begin{verbatim}
abf {
colvars			   example
fullSamples		   2000 
hideJacobian		   on
shared			        	on
sharedfreq		   50000
}
\end{verbatim}

\begin{verbatim}
source eabf.tcl
set eabf_inputname		0
set eabf_outputname	    output.eabf.[myReplica]
set eabf_temperature   	300	
set eabf_outputfreq		20000
\end{verbatim}

It should be noted that in contrast with classical MW--ABF simulations, 
the output files of an MW--eABF simulation only show the results of 
the corresponding replica. One can merge the results, using
{\tt ./eabf.tcl -mergemwabf [merged\_filename] [eabf\_output1] [eabf\_output2] ...}, 
e.g.,
{\tt ./eabf.tcl -mergemwabf merge.eabf eabf.0 eabf.1 eabf.2 eabf.3}.

If one runs an ABF--based calculation, breaking the reaction pathway
into several non--overlapping windows, one can use
{\tt ./eabf.tcl -mergesplitwindow [merge\_gradfilename] [output\_gradfile1] [output\_gradfile2] ...}
to merge the data accrued in these non--overlapping windows. 
This option can be utilized in both eABF and classical ABF simulations, e.g.,
{\tt ./eabf.tcl -mergesplitwindow merge.grad window0.grad window1.grad window2.grad}.

