%%%%%%%%%%%%%%%%%%%%%%%%%%%%%%%%%%%%%%%%%%%%%%%%%%%%%%%%%%%%%%%%%%%%%%%%%%%%
%                                                                          %
%              (C) Copyright 1995 The Board of Trustees of the             %
%                          University of Illinois                          %
%                           All Rights Reserved                            %
%								  	   %
%%%%%%%%%%%%%%%%%%%%%%%%%%%%%%%%%%%%%%%%%%%%%%%%%%%%%%%%%%%%%%%%%%%%%%%%%%%%


\section{Introduction}
\label{section:intro}

\NAMD\ is a parallel molecular dynamics program for UNIX 
platforms designed for high-performance 
simulations in structural biology.  This document describes how to use 
\NAMD, its features, and the platforms on which it runs.
The document is divided into several sections:
\begin{description}
\item[Section \ref{section:intro}] gives an overview of \NAMD.
\item[Section \ref{section:start}] lists the basics for getting started.
\item[Section \ref{section:files}] describes \NAMD\ file formats.
\item[Section \ref{section:psfgen}] explains PSF file generation with psfgen.
\item[Section \ref{section:basic}] lists basic simulation options.
\item[Section \ref{section:add}] lists additional simulation options.
\item[Section \ref{section:xplorequiv}] provides hints for X-PLOR users.
\item[Section \ref{section:sample}] provides sample configuration files.
\item[Section \ref{section:run}] gives details on running \NAMD.
\item[Section \ref{section:avail}] gives details on installing \NAMD.
\end{description}

We have attempted to make this document 
complete and easy to understand and to make \NAMD\ itself
easy to install and run.
We welcome your suggestions for improving the documentation or code
at {\tt namd@ks.uiuc.edu}.

\subsection{New features in version \NAMDVER}

\subsubsection*{Ports to Itanium, Altix, and Opteron/Athlon64/EMT64}

NAMD runs very fast on the Itanium processor under Linux, including
the SGI Altix.  Native binaries for 64-bit x86 processors such as
AMD Opteron and Intel EMT64 are 25\% faster than 32-bit binaries.

\subsubsection*{Port to Mac OS X for Intel Processors}

NAMD has been ported to the new Intel-based Macintosh platform.
Binaries are only 32-bit, but should run on newer 64-bit machines.

\subsubsection*{Ports to Cray XT3 and IBM BlueGene/L (source code only)}

These two large, scalable, but immature platforms are now
supported.  Neither platform supports dynamic linking and our
experience shows that binaries would be quickly out of date, so
only source code is released.  Also, BlueGene/L requires the
latest development version of Charm++, rather than charm-5.9
as shipped with the NAMD source code.

\subsubsection*{Improved Serial Performance, Especially on POWER and PowerPC}

Tuning of the NAMD inner loop has provided a 30\% performance boost
with the IBM XL compiler.  New Mac OS X binaries are up to 70\%
faster, but users must download the IBM XL runtime library from
{\tt http://ftp.software.ibm.com/aix/products/ccpp/vacpp-rte-macos/}

\subsubsection*{Adaptive Biasing Force Free Energy Calculations}

A new method, implemented in Tcl, efficiently calculates the
potential of mean force along a reaction coordinate by applying
adaptive biasing forces to provide uniform sampling.

\subsubsection*{Customizable Replica Exchange Simulations}

The replica exchange method is implemented as a set of Tcl scripts
that use socket connections to drive a set of NAMD jobs, exchanging
temperatures (or any other scriptable parameter) based on energy.

\subsubsection*{Tcl-Based Boundary Potentials}

Tcl-scripted forces may be efficiently applied individually to
large numbers of atoms to implement boundaries and similar forces.

\subsubsection*{Reduced Memory Usage for Unusual Simulations}

Memory usage for simulations of large, highly bonded structures
such as covalent crystals and for sparse simulations such as
coarse-grained models has been greatly reduced without affecting
the performance of typical biopolymer simulations.

\subsubsection*{Support for CHARMM 31 Stream Files and CMAP Crossterms}

Both NAMD and psfgen (standalone or VMD plug) read and interpret
the new CHARMM 31 stream files (combining topology and parameters)
and the new CMAP crossterm (dihedral-dihedral) potential function.

\subsubsection*{Support for OPLS Force Field}

Geometric combining of Lennard-Jones $\sigma$ (radius) parameters
($\sigma_{ij} = \sqrt{\sigma_i \sigma_j}$), as required by the OPLS force
field, is available with the option {\tt vdwGeometricSigma}.


\subsection{\NAMD\ and molecular dynamics simulations}

Molecular dynamics (MD) simulations compute atomic trajectories by solving
equations of motion numerically using empirical force fields, such as the 
CHARMM force field, that approximate the actual atomic force in 
biopolymer systems. Detailed information about MD simulations can be found in
several books such as 
\mycite{(Allen and Tildesley),(McCammon and Harvey)}{ALLE87,MCCA87}. 
In order to conduct MD simulations, various computer programs have been 
developed including
X-PLOR \mycite{(Br\"unger, 1992)}{BRUN92b} and 
CHARMM \mycite{(Brooks \ETAL, 1983)}{BROO83}.
These programs were originally developed for serial machines. 
Simulation of large molecules, however, require enormous computing power. 
One way to achieve such simulations is to utilize parallel computers. In recent 
years, distributed memory parallel computers have been offering
cost-effective computational power.  \NAMD\ was designed to run efficiently
on such parallel 
machines for simulating large molecules. 
\NAMD\ is particularly well suited to the increasingly popular Beowulf-class PC clusters, which are quite similar to the workstation clusters for which is was originally designed.
Future versions of \NAMD\ will also make efficient use of clusters of multi-processor workstations or PCs.
\prettypar
\NAMD\ has several important features: 

\begin{itemize}

\item{\bf Force Field Compatibility}\\
The force field used by \NAMD\ is the same as that used by the programs 
CHARMM \mycite{(Brooks \ETAL, 1983)}{BROO83} and X-PLOR 
\mycite{(Br\"unger, 1992)}{BRUN92b}.  This force field includes local 
interaction terms consisting of bonded interactions between 2, 3, and 4 atoms 
and pairwise interactions including electrostatic and van der Waals forces.
This commonality allows simulations to migrate between these three programs.

\item{\bf Efficient Full Electrostatics Algorithms}\\
\NAMD\ incorporates the Particle Mesh Ewald (PME) algorithm,
which takes the full electrostatic interactions into account.
This algorithm reduces the computational complexity of electrostatic
force evaluation from $O(N^2)$ to $O(N \log N)$.

\item{\bf Multiple Time Stepping}\\
The velocity Verlet integration method
\mycite{M. P. Allen and D. J. Tildesley, 1987}{ALLE87}
is used to advance the positions and velocities of the atoms in time.
To further reduce the cost of the evaluation of 
long-range electrostatic forces, 
a multiple time step scheme is employed.  The local
interactions (bonded, van der Waals and electrostatic interactions within a
specified distance) are calculated at each time step.  The longer range
interactions (electrostatic interactions beyond the specified distance) are
only computed less often.
This amortizes the cost of computing the electrostatic forces over several timesteps.
A smooth splitting function is used to separate a quickly varying short-range portion of the electrostatic interaction from a more slowly varying long-range component.
It is also possible to employ an intermediate timestep for the short-range non-bonded interactions, performing only bonded interactions every timestep.


\item{\bf Input and Output Compatibility}\\
The input and output file formats used by \NAMD\ are identical to those
used by CHARMM and X-PLOR.  Input formats include coordinate files in PDB format
\mycite{(Bernstein \ETAL, 1977)}{BERN77}, structure files in X-PLOR PSF format, 
and energy parameter files in either ~CHARMM or X-PLOR formats.
Output formats include PDB coordinate files and binary DCD trajectory files.
These similarities assure that the molecular dynamics trajectories from \NAMD\ 
can be read by CHARMM or X-PLOR and that the user can exploit the many 
analysis algorithms of the latter packages.

\item{\bf Dynamics Simulation Options}\\
MD simulations may be carried out using several options, including
\begin{itemize}
  \item Constant energy dynamics,
  \item Constant temperature dynamics via
  \begin{itemize}
    \item Velocity rescaling,
    \item Velocity reassignment,
    \item Langevin dynamics,
  \end{itemize}
  \item Periodic boundary conditions,
  \item Constant pressure dynamics via
  \begin{itemize}
    \item Berendsen pressure coupling,
    \item Nos\'{e}-Hoover Langevin piston,
  \end{itemize}
  \item Energy minimization,
  \item Fixed atoms,
  \item Rigid waters,
  \item Rigid bonds to hydrogen,
  \item Harmonic restraints,
  \item Spherical or cylindrical boundary restraints.
\end{itemize}

\item{\bf Easy to Modify and Extend}\\
Another primary design objective for \NAMD\ is extensibility and 
maintainability. In order to achieve this, it is designed in an 
object-oriented style with C++. Since molecular dynamics is a new field,
new algorithms and techniques are continually being developed.
\NAMD's modular design allows one to integrate and test new algorithms 
easily.  If you are contemplating a particular modification to \NAMD\
you are encouraged to contact the developers at {\tt namd@ks.uiuc.edu}
for guidance.

\item{\bf Interactive MD simulations}\\
A system undergoing simulation in \NAMD\ may be viewed and
altered with \VMD; for instance, forces can be applied to a set of atoms
to alter or rearrange part of the molecular structure.  For more information
on \VMD, see {\tt http://www.ks.uiuc.edu/Research/vmd/}.  

\item{\bf Load Balancing}\\
An important factor in parallel applications is the equal distribution
of computational load among the processors. In parallel molecular simulation,
a spatial decomposition that evenly distributes the computational load
causes the region of space mapped to each processor to become very irregular, 
hard to compute and difficult to generalize to the evaluation of many different
types of forces.  \NAMD\ addresses this problem by using a simple uniform 
spatial decomposition where the entire model is split into uniform cubes of 
space called {\em patches}. An initial load balancer assigns patches
and the calculation of interactions among the atoms within them
to processors such that the computational load is balanced as much as possible.
During the simulation, an incremental load balancer monitors the load
and performs necessary adjustments.

\end{itemize}

\subsection{User feedback}

If you have problems installing or running \NAMD\ after
reading this document, please send a
complete description of the problem by email to {\tt namd@ks.uiuc.edu}.  If
you discover and fix a problem not described in this manual we would
appreciate if you would tell us about this as well, so we can alert
other users and incorporate the fix into the public distribution.
\prettypar
We are interested in making \NAMD\ more useful to the molecular modeling
community.  Your suggestions are welcome at {\tt namd@ks.uiuc.edu}.
We also appreciate hearing about how you are using \NAMD\ in your work.

\subsection{Acknowledgments}

This work is supported by grants from the National Science
Foundation (BIR-9318159) and the National Institute of Health 
(PHS 5 P41 RR05969-04).
\prettypar
The authors would particularly like to thank the members of the
Theoretical Biophysics Group, past and present, who have helped
tremendously in making suggestions, pushing for new features, and
testing bug-ridden code.

