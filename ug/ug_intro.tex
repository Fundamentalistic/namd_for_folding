%%%%%%%%%%%%%%%%%%%%%%%%%%%%%%%%%%%%%%%%%%%%%%%%%%%%%%%%%%%%%%%%%%%%%%%%%%%%
%                                                                          %
%              (C) Copyright 1995 The Board of Trustees of the             %
%                          University of Illinois                          %
%                           All Rights Reserved                            %
%								  	   %
%%%%%%%%%%%%%%%%%%%%%%%%%%%%%%%%%%%%%%%%%%%%%%%%%%%%%%%%%%%%%%%%%%%%%%%%%%%%

%%%%%%%%%%%%%%%%%%%%%%%%%%%%%%%%%%%%%%%%%%%%%%%%%%%%%%%%%%%%%%%%%%%%%%%%%%%%
% RCS INFORMATION:
%
%       $RCSfile: ug_intro.tex,v $
%       $Author: jim $        $Locker:  $                $State: Exp $
%       $Revision: 1.10 $      $Date: 1999/10/06 21:06:00 $
%
%%%%%%%%%%%%%%%%%%%%%%%%%%%%%%%%%%%%%%%%%%%%%%%%%%%%%%%%%%%%%%%%%%%%%%%%%%%%
% DESCRIPTION:
%	This is the introduction to the namd users guide
%
%%%%%%%%%%%%%%%%%%%%%%%%%%%%%%%%%%%%%%%%%%%%%%%%%%%%%%%%%%%%%%%%%%%%%%%%%%%%
% REVISION HISTORY:
%
% $Log: ug_intro.tex,v $
% Revision 1.10  1999/10/06 21:06:00  jim
% Various doc updates for 2.1.
%
% Revision 1.9  1999/03/18 02:16:49  jim
% Spelling fixes.
%
% Revision 1.8  1999/03/18 01:49:51  jim
% Many changes, now mostly ready for 2.0 release.
%
% Revision 1.7  1999/03/09 20:58:47  jim
% Changed verb to tt in most places to help out latex2html.
%
% Revision 1.6  1999/02/23 02:01:13  jim
% Small tweaks.
%
% Revision 1.5  1999/02/03 20:44:59  jim
% Fixed several configuration options.
%
% Revision 1.4  1998/03/06 21:06:31  jim
% Put temperature couple back in.
%
% Revision 1.3  1998/03/06 11:03:29  jim
% Added nonbondedFreq parameter.
%
% Revision 1.2  1998/03/06 09:08:10  jim
% Documented new features in 2.0.  Added authors.  Still needs converse docs.
%
% Revision 1.1  1998/01/05 21:12:32  dhardy
% user guide, first draft
%
% Revision 1.10  1997/08/07 15:54:32  dhardy
% edit presentation style
%
% Revision 1.9  1997/06/19 19:26:12  nealk
% Minor text corrections.
%
% Revision 1.8  1996/06/06 18:02:01  dalke
% fixed typos for 1.4 release
%
% Revision 1.7  1996/05/15  19:21:07  jean
% ready (I hope) for 1.4 beta release
%
% Revision 1.6  1996/02/04 20:34:02  jean
% Various grammer, spelling, punctuation fixes.
%
% Revision 1.5  1995/06/30 15:42:44  gursoy
% fixed references
%
% Revision 1.4  95/06/28  23:55:15  23:55:15  gursoy (Attila Gursoy)
% worked on teh format a little bit
% 
% Revision 1.3  95/06/26  15:52:31  15:52:31  gursoy (Attila Gursoy)
%  load balancing paragraph still needs some work
% 
% Revision 1.2  95/06/22  17:09:02  17:09:02  gursoy (Attila Gursoy)
% First attempt for introduction. still working on it.
% 
% Revision 1.1  95/06/21  16:28:13  16:28:13  nelson (Mark T. Nelson)
% Initial revision
% 
%%%%%%%%%%%%%%%%%%%%%%%%%%%%%%%%%%%%%%%%%%%%%%%%%%%%%%%%%%%%%%%%%%%%%%%%%%%%


\section{Introduction}
\label{section:intro}

\NAMD\ is a parallel molecular dynamics program for UNIX 
platforms designed for high-performance 
simulations in structural biology.  This document describes how to use 
\NAMD, its features, and the platforms on which it runs.
The document is divided into several sections:
\begin{description}
\item[Section \ref{section:intro}] gives an overview of \NAMD.
\item[Section \ref{section:start}] lists the basics for getting started.
\item[Section \ref{section:files}] describes \NAMD\ file formats.
\item[Section \ref{section:basic}] lists basic simulation options.
\item[Section \ref{section:add}] lists additional simulation options.
\item[Section \ref{section:xplorequiv}] provides hints for X-PLOR users.
\item[Section \ref{section:sample}] provides sample configuration files.
\item[Section \ref{section:run}] gives details on running \NAMD.
\item[Section \ref{section:avail}] gives details on installing \NAMD.
\end{description}

We have attempted to make this document and the \PG\ both
complete and easy to understand and to make \NAMD\ itself
easy to install and run.  Please take a moment to help us improve
the documentation and the code by filling out the users' survey
at the end of this section.

\subsection{New features in version \NAMDVER}

\subsubsection*{Mollified impulse method (MOLLY)}

This method eliminates the components of the long range electrostatic
forces which contribute to resonance along bonds to hydrogen atoms,
allowing a fullElectFrequency of 6 (vs.\ 4) with a 1~fs timestep
without using ``rigidBonds all''.  You may use ``rigidBonds water'' but
using ``rigidBonds all'' with MOLLY makes no sense since the degrees of
freedom which MOLLY protects from resonance are already frozen.

\subsubsection*{Non-orthogonal periodic cells}

The parameters (cellBasisVector[123] and cellOrigin) have not changed,
but the basis vectors may now be in any direction.

\subsubsection*{Interactive molecular dynamics}

NAMD now works directly with VMD ({\tt http://www.ks.uiuc.edu/Research/vmd/})
to allow you to view and interactively steer your simulation.  NAMD will
wait for VMD to connect on startup.

\subsubsection*{Faster particle mesh Ewald}

If for some reason you wish to use the slower DPME implementation
of the particle mesh Ewald method add ``useDPME yes''.  If you set
cellOrigin to something other than (0,0,0) the energy may differ
slightly between the old and new implementations.  For even better
performance get the NAMD source code and compile with FFTW.

\subsection{\NAMD\ and molecular dynamics simulations}

Molecular dynamics (MD) simulations compute atomic trajectories by solving
equations of motion numerically using empirical force fields, such as the 
CHARMM force field, that approximate the actual atomic force in 
biopolymer systems. Detailed information about MD simulations can be found in
several books such as 
\mycite{(Allen and Tildesley),(McCammon and Harvey)}{ALLE87,MCCA87}. 
In order to conduct MD simulations, various computer programs have been 
developed including
X-PLOR \mycite{(Br\"unger, 1992)}{BRUN92b} and 
CHARMM \mycite{(Brooks \ETAL, 1983)}{BROO83}.
These programs were originally developed for serial machines. 
Simulation of large molecules, however, require enormous computing power. 
One way to achieve such simulations is to utilize parallel computers. In recent 
years, distributed memory parallel computers have been offering
cost-effective computational power.  \NAMD\ was designed to run efficiently
on such parallel 
machines for simulating large molecules. 
\NAMD\ is particularly well suited to the increasingly popular Beowulf-class PC clusters, which are quite similar to the workstation clusters for which is was originally designed.
Future versions of \NAMD\ will also make efficient use of clusters of multi-processor workstations or PCs.
\prettypar
\NAMD\ has several important features: 

\begin{itemize}

\item{\bf Force Field Compatibility}\\
The force field used by \NAMD\ is the same as that used by the programs 
CHARMM \mycite{(Brooks \ETAL, 1983)}{BROO83} and X-PLOR 
\mycite{(Br\"unger, 1992)}{BRUN92b}.  This force field includes local 
interaction terms consisting of bonded interactions between 2, 3, and 4 atoms 
and pairwise interactions including electrostatic and van der Waals forces.
This commonality allows simulations to migrate between these three programs.

\item{\bf Efficient Full Electrostatics Algorithms}\\
\NAMD\ incorporates the Distributed Parallel Multipole Tree Algorithm 
(DPMTA) \mycite{(Board \ETAL)}{duke:fma} and Distributed Particle Mesh
Ewald (DPME) algorithms, which takes the full electrostatic
interactions into account.  These algorithm reduces the computational
complexity of electrostatic force evaluation from $O(N^2)$ to $O(N)$
or $O(N \log N)$.

\item{\bf Multiple Time Stepping}\\
The velocity Verlet integration method
\mycite{M. P. Allen and D. J. Tildesley, 1987}{ALLE87}
is used to advance the positions and velocities of the atoms in time.
To further reduce the cost of the evaluation of 
long-range electrostatic forces, 
a multiple time step scheme is employed.  The local
interactions (bonded, van der Waals and electrostatic interactions within a
specified distance) are calculated at each time step.  The longer range
interactions (electrostatic interactions beyond the specified distance) are
only computed less often.
This amortizes the cost of computing the electrostatic forces over several timesteps.
A smooth splitting function is used to separate a quickly varying short-range portion of the electrostatic interaction from a more slowly varying long-range component.
It is also possible to employ an intermediate timestep for the short-range non-bonded interactions, performing only bonded interactions every timestep.


\item{\bf Input and Output Compatibility}\\
The input and output file formats used by \NAMD\ are identical to those
used by CHARMM and X-PLOR.  Input formats include coordinate files in PDB format
\mycite{(Bernstein \ETAL, 1977)}{BERN77}, structure files in X-PLOR PSF format, 
and energy parameter files in either ~CHARMM or X-PLOR formats.
Output formats include PDB coordinate files and binary DCD trajectory files.
These similarities assure that the molecular dynamics trajectories from \NAMD\ 
can be read by CHARMM or X-PLOR and that the user can exploit the many 
analysis algorithms of the latter packages.

\item{\bf Dynamics Simulation Options}\\
MD simulations may be carried out using several options, including
\begin{itemize}
  \item Constant energy dynamics,
  \item Constant temperature dynamics via
  \begin{itemize}
    \item Velocity rescaling,
    \item Velocity reassignment,
    \item Langevin dynamics,
  \end{itemize}
  \item Periodic boundary conditions,
  \item Constant pressure dynamics via
  \begin{itemize}
    \item Berendsen pressure coupling,
    \item Nos\'{e}-Hoover Langevin piston,
  \end{itemize}
  \item Energy minimization,
  \item Fixed atoms,
  \item Rigid waters,
  \item Rigid bonds to hydrogen,
  \item Harmonic restraints,
  \item Spherical or cylindrical boundary restraints.
\end{itemize}

\item{\bf Easy to Modify and Extend}\\
Another primary design objective for \NAMD\ is extensibility and 
maintainability. In order to achieve this, it is designed in an 
object-oriented style with C++. Since molecular dynamics is a new field,
new algorithms and techniques are continually being developed.
\NAMD's modular design allows one to integrate and test new algorithms 
easily.  The structure and design of the program 
is described in the \PG\ with sufficient
detail to allow additions of such
new algorithms.

\item{\bf Interactive MD simulations}\\
A system undergoing simulation in \NAMD\ may be viewed and
altered with \VMD; for instance, forces can be applied to a set of atoms
to alter or rearrange part of the molecular structure.  For more information
on \VMD, see {\tt http://www.ks.uiuc.edu/Research/vmd/}.  

\item{\bf Load Balancing}\\
An important factor in parallel applications is the equal distribution
of computational load among the processors. In parallel molecular simulation,
a spatial decomposition that evenly distributes the computational load
causes the region of space mapped to each processor to become very irregular, 
hard to compute and difficult to generalize to the evaluation of many different
types of forces.  \NAMD\ addresses this problem by using a simple uniform 
spatial decomposition where the entire model is split into uniform cubes of 
space called {\em patches}. An initial load balancer assigns patches
to processors such that the computational load is balanced as much as possible.
During the simulation, an incremental load balancer monitors the load
and performs necessary adjustments.

\end{itemize}

\subsection{User feedback}

If you have problems installing or running \NAMD\ after
reading this document, please send a
complete description of the problem by email to {\tt namd@ks.uiuc.edu}.  If
you discover and fix a problem not described in this manual or the
\PG, we would
appreciate if you would tell us about this as well, so we can alert
other users and incorporate the fix into the public distribution.
\prettypar
We are interested in making \NAMD\ more useful to the molecular modeling
community.  Please take a few minutes to complete the short survey on
the next page and mail or fax it to

\bigskip

\NAMD\ Support

Theoretical Biophysics Group

Beckman Institute

University of Illinois

405 N. Mathews St.

Urbana, IL  61801

FAX (217) 244-6078

\bigskip

\noindent or email the relevant information
to {\tt namd@ks.uiuc.edu} with subject ``Survey.''  
This will help
us make \NAMD\ relevant and responsive to the needs of our user
community.

\newpage
\begin{enumerate}

\item Name:

\item Company/University:

\item Address:

\kern 0.4in

\item Email address:

\item Research area:

\kern 0.3in

\item Molecular systems to be studied with \NAMD:

\kern 0.4in

\item \NAMD\ features most useful for your research:

\kern 0.4in

\item I run \NAMD\ on the following platforms and operating systems:

\kern 0.5in

\item I would like to run \NAMD\ on:

\kern 0.2in

\item How can we improve the \NAMD\ documentation?

\kern 1.2in

\item Other comments:

\end{enumerate}
\newpage

\subsection{Acknowledgments}

This work is supported by grants from the National Science
Foundation (BIR-9318159) and the National Institute of Health 
(PHS 5 P41 RR05969-04).
\prettypar
The authors would particularly like to thank the members of the
Theoretical Biophysics Group, past and present, who have helped
tremendously in making suggestions, pushing for new features, and
testing bug-ridden code.

