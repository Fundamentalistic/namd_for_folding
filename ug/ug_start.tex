
\section{Getting Started}
\label{section:start}

\subsection{What is needed}

Before running \NAMD, explained in section \ref{section:run}, 
the following are be needed:
\begin{itemize}
\item A CHARMM force field in either CHARMM or X-PLOR format.
\item
An X-PLOR format PSF file describing the molecular structure.
\item
The initial coordinates of the molecular system in the form of a PDB file.  
\item
A \NAMD\ configuration file.
\end{itemize}

\NAMD\ provides the \verb#psfgen# utility,
documented in Section~\ref{section:psfgen},
which is capable of generating the required PSF and PDB files by
merging PDB files and guessing coordinates for missing atoms.
If \verb#psfgen# is insufficient for your system,
we recommend that you obtain access to either CHARMM or X-PLOR,
both of which are capable of generating the required files.

\subsection{\NAMD\ configuration file}
\label{section:config}

Besides these input and output files, \NAMD\ also uses 
a file referred to as the {\it configuration file\/}.  
This file specifies what dynamics options and values that 
\NAMD\ should use, such as the number of timesteps to perform, 
initial temperature, etc.  
The options and values in this file control how 
the system will be simulated.  
The NAMD configuration file is specified on the NAMD command line,
either before or after the various parallel execution options described
in section \ref{section:run}.

A \NAMD\ configuration file contains a set of options and values.  
The options and values specified determine the exact behavior of
\NAMD, what features are active or inactive, how long the simulation
should continue, etc.  Section \ref{section:configsyntax} describes how
options are specified within a \NAMD\ configuration file.  Section
\ref{section:requiredparams} lists the parameters which are required
to run a basic simulation.  Section \ref{section:xplorequiv}
describes the relation between specific \NAMD\ and X-PLOR dynamics
options.  Several sample \NAMD\ configuration files are shown
in section \ref{section:sample}.

During execution NAMD will change to the directory containing the
configuration file so that all file paths in the configuration file
are relative to the configuration file directory.
Multiple configuration files may be specified on the command line
and the will be read in order, but all file paths will be relative
to the first configuration file to call a ``run'' (or ``minimize'' or ``startup'') command, or
to the last configuration file if ``run'' is not called.

Commands or parameters may also be specified directly on the
command line via \verb#--keyword value# argument pairs, for example
\verb#--outputenergies 100# \verb#--run 100# \verb#-- checkpoint#.
This may be used to include multiple configuration files without altering the
working directory via \verb#--source /path/to/second.conf#.
Note that escaping or quoting of command line parameter values
containing spaces may be difficult or impossible on some systems due
to multiple levels of scripts called during the NAMD parallel launch process
and because the keyword and value are simply merged into a single string
that is passed to the Tcl interpreter.

If the argument \verb#--tclmain# is present, all following arguments will
be passed to the Tcl interpreter as a script file and arguments accessible
via the standard argc and argv variables.
Note that Charm++ arguments such as \verb#+pemap# are processed during
Charm++ startup and will not be passed to Tcl.

If the first argument is \verb#+tclsh#,
Charm++ argument parsing and startup are not performed,
the Tcl interpreter is initialized without NAMD scripting features,
and all following arguments are passed to Tcl.
Statically linked packages such as psfgen are available via ``package require ...''.

\subsubsection{Configuration parameter syntax}
\label{section:configsyntax}
Each line
in the configuration files consists of a $keyword$ identifying the option
being specified, and a $value$ which is a parameter to be used for this
option.  The keyword and value can be separated by only white space:
\begin{verbatim}
keyword            value
\end{verbatim}
or the keyword and value can be separated by an equal sign and white space:
\begin{verbatim}
keyword      =     value
\end{verbatim}
Blank lines in the configuration file are ignored.  Comments are prefaced by
a \verb!#! and may appear on the end of a line with actual values:
\begin{verbatim}
keyword            value          #  This is a comment
\end{verbatim}
or may be at the beginning of a line:
\begin{verbatim}
#  This entire line is a comment . . . 
\end{verbatim}
Some keywords require several lines of data.
These are generally implemented to either allow the data to be read from a file:
\begin{verbatim}
keyword            filename
\end{verbatim}
or to be included inline using Tcl-style braces:
\begin{verbatim}
keyword {
  lots of data
}
\end{verbatim}

The specification of the keywords is case insensitive 
so that any combination of 
upper and lower case letters will have the same meaning.  
Hence, {\tt DCDfile} and {\tt dcdfile} 
are equivalent.  The capitalization in the values, however, may be important.
Some values indicate file names, in which capitalization is critical.  
Other values such as {\tt on} or {\tt off} are case insensitive.

\subsubsection{Tcl scripting interface and features}
\label{section:tclscripting}

When compiled with Tcl (all released binaries) the config file
is parsed by Tcl in a fully backwards compatible manner with the
added bonus that any Tcl command may also be used.  This alone allows:
\begin{itemize}
 \item the ``\icommand{source}'' command to include other files (works w/o Tcl too!),
 \item the ``\icommand{print}'' command to display messages (``puts'' to stdout fails on some platforms),
 \item environment variables through the env array (``\$env(USER)''), and
 \item user-defined variables (``set base sim23'', ``dcdfile \${base}.dcd'').
\end{itemize}

Additional features include:
\begin{itemize}
 \item The ``\icommand{callback}'' command takes a 2-parameter Tcl procedure which
    is then called with a list of labels and a list of values during
    every timestep, allowing analysis, formatting, whatever.
 \item The ``\icommand{run}'' command takes a number of steps to run (overriding the
    now optional numsteps parameter, which defaults to 0) and can be
    called repeatedly.  You can ``run 0'' just to get energies.
    Normally the preceeding timestep is repeated to account for any modifications to the
    energy function; this can be avoided with ``\icommand{run norepeat}''.
 \item The ``\icommand{minimize}'' command is similar to ``run'' and performs
    minimization for the specified number of force evaluations.
 \item The ``\icommand{startup}'' command will trigger simulation startup
    as would the first ``run'' or ``minimize'' command, but without any force/energy evaluation.
 \item Configuration file parameter introspection is supported by
   invoking a (case-insensitive) parameter keyword with no argument (e.g., ``numsteps'')
   and by the helper commands ``\icommand{isset}'' and ``\icommand{istrue}''.
   Note that keywords are not parsed until the first ``run'' command, and
   before this values are treated as unformatted strings, so for example
   ``eFieldOn'' and ``eField'' may return ``yes'' and ``1 2 3'' before
   the first ``run'' command, but ``1'' and ``1.0 2.0 3.0'' after parsing
   (``istrue eFieldOn'' would return ``1'' in both cases).
   Similarly, ``isset badparam'' will return ``0'' before parsing
   but raise an ``unknown parameter'' error after.
 \item Between ``run'' commands the \iparam{reassignTemp},
    \iparam{rescaleTemp}, and
    \iparam{langevinTemp} parameters can be changed to allow simulated
    annealing protocols within a single config file.
    The \iparam{useGroupPressure}, \iparam{useFlexibleCell},
    \iparam{useConstantArea}, \iparam{useConstantRatio},
    \iparam{LangevinPiston}, \iparam{LangevinPistonTarget},
    \iparam{LangevinPistonPeriod}, \iparam{LangevinPistonDecay},
    \iparam{LangevinPistonTemp}, \iparam{SurfaceTensionTarget},
    \iparam{BerendsenPressure}, \iparam{BerendsenPressureTarget},
    \iparam{BerendsenPressureCompressibility}, and
    \iparam{BerendsenPressureRelaxationTime}
    parameters may be changed to allow pressure equilibration.
    The \iparam{fixedAtoms}, \iparam{constraintScaling}, and
    \iparam{nonbondedScaling} parameters may be
    changed to preserve macromolecular conformation during minimization and
    equilibration (fixedAtoms may only be disabled, and requires that
    \iparam{fixedAtomsForces} is enabled to do this).
    The \iparam{consForceScaling} parameter may be changed to vary steering forces
    or to implement a time-varying electric field that affects specific atoms.
    The \iparam{eField}, \iparam{eFieldFreq}, and
    \iparam{eFieldPhase} parameters may be changed to implement
    at time-varying electric field that affects all atoms.
    The \iparam{alchLambda} and \iparam{alchLambda2}
    parameters may be changed during alchemical free energy runs.
    The \iparam{DCDfile} may be changed to write binary coordinate trajectory
    output to separate files.
    The \iparam{restartname} may be changed to write restart output to separate files.
 \item The ``\icommand{checkpoint}'' and ``revert'' commands (no arguments) allow
    a scripted simulation to save and restore (in memory) to a single prior state.
    The ``output'' and ``reinitatoms'' commands support multiple saved states using files.
    Multiple saved states in memory are supported by the commands
    ``\icommand{checkpointStore}'', ``\icommand{checkpointLoad}'', ``\icommand{checkpointSwap}'', and
    ``\icommand{checkpointFree}'', all of which take a string key as an argument,
    plus an optional second argument that is either
    replica index (the checkpoint is stored asynchronously on the target replica)
    or the keyword ``global'' (the target replica is computed as a hash of the key).
 \item The ``\icommand{output}'' command takes an output file basename and causes
    .coor, .vel, and .xsc files to be written with that name.
    Alternatively, ``\icommand{output withforces}'' and
    ``\icommand{output onlyforces}'' will write a .force file
    either in addition to or instead of the regular files.
 \item The ``\icommand{reinitatoms}'' command reinitializes coordinates,
    velocities, and periodic cell dimensions to those initially read in
    (random velocities are generated if they were not read from a file).
    An optional file basename argument (matching that passed to the \icommand{output} command)
    causes .coor, .vel, and .xsc files to be read,
    assuming the format indicated by the \iparam{binaryoutput} parameter.
 \item The ``\icommand{move}'' command repositions individual atoms, including fixed atoms.
    Arguments are a 1-based atom ID, ``to'' or ``by'', and a list of three numbers,
    e.g., ``move 1 by \{0.4 0.2 -0.1\}''.
    Atoms may not be moved by more than a single patch between ``run'' commands.
 \item The ``\icommand{exit}'' command writes output files and exits cleanly.
 \item The ``\icommand{abort}'' command concatenates its arguments into
    an error message and exits immediately without writing output files.
 \item The ``\icommand{numPes}'', ``\icommand{numNodes}'', and
    ``\icommand{numPhysicalNodes}'' commands allow performance-tuning
    parameters to be set based on the parallel execution environment.
 \item The ``\icommand{reinitvels}'' command reinitializes velocities to a
    random distribution based on the given temperature.
 \item The ``\icommand{rescalevels}'' command rescales velocities by
    the given factor.
 \item The ``\icommand{reloadCharges}'' command reads new atomic charges from
    the given file, which should contain one number for each atom, separated
    by spaces and/or line breaks.
 \item The ``\icommand{consForceConfig}'' command takes a list of
    0-based atom indices and a list of forces which replace the existing
    set of constant forces (\iparam{constantForce} must be on).
 \item The ``\icommand{measure}'' command allows user-programmed calculations to
    be executed in order to facilitate automated methods.  (For
    example, to revert or change a parameter.)  A number of measure
    commands are included in the NAMD binary; the module has been designed
    to make it easy for users to add additional measure commands.  
 \item The ``\icommand{coorfile}'' command allows NAMD to perform force and energy
    analysis on trajectory files.  ``coorfile open dcd {\tt filename}'' opens
    the specified DCD file for reading.  ``coorfile read'' reads the next
    frame in the opened DCD file, replacing NAMD's atom coordinates with the
    coordinates in the frame, and returns 0 if successful or -1 if  
    end-of-file was reached.  ``coorfile skip'' skips past one frame in the
    DCD file; this is significantly faster than reading coordinates and 
    throwing them away.  ``coorfile close'' closes the file.   
    The ``coorfile'' command is not available on the Cray T3E.

    Force and energy analysis are especially useful in the context of 
    pair interaction calculations; see Sec.~\ref{section:pairinteraction}
    for details, as well as the example scripts in Sec.~\ref{section:sample}.
\end{itemize}


Please note that while NAMD has traditionally allowed comments to be
started by a \# appearing anywhere on a line, Tcl only allows comments
to appear where a new statement could begin.  With Tcl config file
parsing enabled (all shipped binaries) both NAMD and Tcl comments are
allowed before the first ``run'' command.  At this point only pure Tcl
syntax is allowed.  In addition, the ``;\#'' idiom for Tcl comments will
only work with Tcl enabled.  NAMD has also traditionally allowed
parameters to be specified as ``param=value''.  This is supported, but
only before the first ``run'' command.  Some examples:

\begin{verbatim}
# this is my config file                            <- OK
reassignFreq 100 ; # how often to reset velocities  <- only w/ Tcl
reassignTemp 20 # temp to reset velocities to       <- OK before "run"
run 1000                                            <- now Tcl only
reassignTemp 40 ; # temp to reset velocities to     <- ";" is required
\end{verbatim}

NAMD has also traditionally allowed parameters to be specified as
``param=value'' as well as ``param value''.  This is supported, but only
before the first ``run'' command.  For an easy life, use ``param value''.


\subsubsection{Multiple-copy/replica-exchange scripting interface}
\label{section:replicascripting}

Multiple-copy (or replica-based) algorithms are supported by the following commands,
which utilize two-sided semantics modeled on MPI:

\begin{itemize}
  \item \icommand{myReplica}
  \item \icommand{numReplicas}
  \item \icommand{replicaBarrier}
  \item \icommand{replicaSend} {\em data} {\em dest}
  \item \icommand{replicaRecv} {\em source}
  \item \icommand{replicaSendrecv} {\em data} {\em dest} {\em source}
  \item \icommand{replicaAtomSend} {\em dest}
  \item \icommand{replicaAtomRecv} {\em source}
  \item \icommand{replicaAtomSendrecv} {\em dest} {\em source}
\end{itemize}

The replicaSend/Sendrecv {\em data} argument may be any string,
and hence any Tcl object (e.g., a list) that can be represented as a string.
Data received from the {\em source} replica is returned by replicaRecv/Sendrecv.
In order to ensure message ordering, replicaSend/Sendrecv will block
until the corresponding remote receive call (except when replicaSend
is called from inside replicaEval, as discussed below).

The parameter {\iparam{replicaUniformPatchGrids}} must be true for 
atom exchange (replicaAtom...) or remote checkpointing (checkpoint... with a second argument, see below).

The following additional commands utilize one-sided semantics,
and should provide a complete feature set for running a simulation with
fewer NAMD replica partitions than logical replicas:

\begin{itemize}
  \item \icommand{checkpointStore} {\em key} ?{\em replica} or global?
  \item \icommand{checkpointLoad} {\em key} ?{\em replica} or global?
  \item \icommand{checkpointSwap} {\em key} ?{\em replica} or global?
  \item \icommand{checkpointFree} {\em key} ?{\em replica} or global?
  \item \icommand{replicaEval} {\em replica} {\em script}
  \item \icommand{replicaYield} ?{\em seconds}?
  \item \icommand{replicaDcdFile} {\em index}|off ?{\em filename}?
\end{itemize}

The {\em key} can be any string.
By default the checkpoint is stored in the memory of the replica the command is called on.
If you specify a replica index the checkpoint is stored asynchronously in that replica's memory.
If you specify ``global'' a hash is computed based on the key to select the replica on which
to store the checkpoint.
You can have checkpoints with the same key stored on multiple replicas at once if you really want to.
The checkpoint... commands will not return until the checkpoint operation has completed.

Storing checkpoints is not atomic.
If two replicas try to store a checkpoint with the same key on the same replica at the same
time you may end up with a mix of the two (and probably duplicate/missing atoms).
If one replica tries to load a checkpoint while another replica is storing it the same may happen.
You cannot store a checkpoint on a replica until that replica has created its own patch data structures.
This can be guaranteed by calling ``startup'' and ``replicaBarrier'' before any remote checkpoint calls.

The replicaEval command asynchronously executes its script in the top-level context
of the target replica's Tcl interpreter and returns the result or error.
This should be general enough to build any kind of work scheduler or shared data structure you need.
If you want to call replicaEval repeatedly, e.g., to check if some value has been set,
you should call ``replicaYield {\em seconds}'' in between, as this will introduce a delay but
still enable processing of asynchronous calls from other replicas.
Potentially blocking functions such as replicaRecv should not be called from within replicaEval,
nor should functions such as run, checkpointLoad/Store, and replicaAtomSend/Recv that would
require the simulation of the remote replica to be halted.
It is allowed to call replicaSend (but not replicaSendrecv) from within replicaEval, since
replicaSend is non-blocking and one-sided (but potentially overtaking) in this context.
Rather than polling a remote replica (e.g., for work) via replicaEval, it is more efficient
to register a request via replicaEval and then call replicaRecv to wait for notification.

The replicaDcdFile command is similar to the \icommand{dcdFile} command in that it changes the
trajectory output file, but the file is actually opened by a different replica partition
and may be written to by any other partition that calls replicaDcdFile with the same index
but no filename argument.  If a filename argument is given, any file currently associated
with the index is closed and a new file created, even if the new and old filenames are the same.
The new file is created only when the next trajectory frame is written,
not during the replicaDcdFile command itself.
The caller must ensure that an index is not used before it is associated with a filename,
and that each index is in use by only one replica at a time.
The keyword ``off'' will return to writing the local trajectory file set by the dcdFile command.


\subsubsection{Python scripting interface and features}
\label{section:pythonscripting}

NAMD may be compiled with an embedded Python interpreter
via the config script option \verb$--$with\verb$-$python.
The default embedded Tcl interpreter is also required to enable Python scripting.
Released NAMD binaries do not support Python scripting at this time
due to portability issues with the extensive Python library.

Python scripting is accessed via the Tcl ``\icommand{python}'' command,
which functions in either expression mode or script mode.
When passed a single-line string,
the Python interpreter will evaluate the expression in the string and return the result.
Python containers (tuples, lists, etc.) are converted to Tcl lists
and all other values are converted to their string representation,
which is typically compatible with Tcl.
For example, ``[python ( 1 + 1, 'abc' + '123' )]'' evaluates to the Tcl list ``2 abc123''.

When the python command is passed a single multi-line string (typically enclosed in braces),
the Python interpreter will execute the code in the string and return nothing.
Because of Python's indentation-sensitive syntax the enclosed code can not be indented.

Calls to Tcl from Python code are supported by the tcl module functions
\icommand{tcl.call()}, which takes the Tcl command name and its arguments as separate arguments
and performs limited container and string conversions as described above,
and \icommand{tcl.eval()}, which passes a single string unmodified to the Tcl interpreter.
Both functions return the result as a string, so numerical results must be explicitly
cast to float or int as appropriate.

NAMD simulation parameters and commands are wrapped for convenience by the ``namd'' object.
Any NAMD simulation parameter may be set by assigning to the corresponding case-insensitive
attribute of the namd object, e.g., ``namd.timestep = 1.0'', and similarly
read (as a string) by access, e.g., ``ts = float(namd.TimeStep)''.
Assignment corresponds exactly to normal config file parsing, i.e., ``timestep 1.0'',
and hence multiple assignment will generate an error just as would repeated parameters.
For convenience, multiple parameters may be set at once by passing them as keyword arguments,
e.g., ``namd(langevin=True, langevinDamping=5., langevinTemp=100.)''.
NAMD (and other) commands in the Tcl interpreter may be called as a method of the namd object,
e.g., ``namd.run(1000)'' and ``namd.output('myfile')''.

The NAMD 1-4scaling parameter is incompatible with Python syntax, but may be accessed
several other ways, e.g., ``namd.param('1-4scaling',1.0)'', ``tcl.call('1-4scaling',1.0)'',
or ``tcl.eval('1-4scaling 1.0')''.

The following example illustrates various aspects of the Python scripting interface:
\begin{verbatim}
set a 1
cutoff 12.0
python {
# do not indent
namd.pairlistDist = float(namd.Cutoff) + float(tcl.eval("set a")) # cast strings to float
b = 2
namd(switching=True, switchdist = float(namd.cutoff) - b) # case insensitive
}
set c [python $a + b]
\end{verbatim}

\subsubsection{Required \NAMD\ configuration parameters}
\label{section:requiredparams}

The following parameters are {\em required} for every
\NAMD\ simulation:

\begin{itemize}

\item
{\tt numsteps} (page \pageref{param:numsteps}),

\item
{\tt coordinates} (page \pageref{param:coordinates}),

\item
{\tt structure} (page \pageref{param:structure}),

\item
{\tt parameters} (page \pageref{param:parameters}),

\item
{\tt exclude} (page \pageref{param:exclude}), 

\item
{\tt outputname} (page \pageref{param:outputname}), 

\item
one of the following three:
\begin{itemize}
\item
{\tt temperature} (page \pageref{param:temperature}),

\item
{\tt velocities} (page \pageref{param:velocities}),

\item
{\tt binvelocities} (page \pageref{param:binvelocities}).
\end{itemize}

\end{itemize}

\noindent These required parameters specify the most basic properties of
the simulation.  %  that is to be performed.
In addition, it is highly recommended that 
{\tt pairlistdist} be specified with a 
value at least one greater than {\tt cutoff}.

