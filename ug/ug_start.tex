
\section{Getting Started}
\label{section:start}

\subsection{What is needed}

Before running \NAMD, explained in section \ref{section:run}, 
the following are be needed:
\begin{itemize}
\item A CHARMM force field in either CHARMM or X-PLOR format.
\item
An X-PLOR format PSF file describing the molecular structure.
\item
The initial coordinates of the molecular system in the form of a PDB file.  
\item
A \NAMD\ configuration file.
\end{itemize}

We strongly recommend that you have access to either CHARMM or X-PLOR,
either of which is capable of generating both the PSF and PDB files.
\NAMD\ currently provides no automatic method of generating these files.

\subsection{\NAMD\ configuration file}
\label{section:config}

Besides these input and output files, \NAMD\ also uses 
a file referred to as the {\it configuration file\/}.  
This file specifies what dynamics options and values that 
\NAMD\ should use, such as the number of timesteps to perform, 
initial temperature, etc.  
The options and values in this file control how 
the system will be simulated.  

A \NAMD\ configuration file contains a set of options and values.  
The options and values specified determine the exact behavior of
\NAMD, what features are active or inactive, how long the simulation
should continue, etc.  Section \ref{section:configsyntax} describes how
options are specified within a \NAMD\ configuration file.  Section
\ref{section:requiredparams} lists the parameters which are required
to run a basic simulation.  Section \ref{section:xplorequiv}
describes the relation between specific \NAMD\ and X-PLOR dynamics
options.  Several sample \NAMD\ configuration files are shown
in section \ref{section:sample}.


\subsubsection{Configuration parameter syntax}
\label{section:configsyntax}
Each line
in the configuration files consists of a $keyword$ identifying the option
being specified, and a $value$ which is a parameter to be used for this
option.  The keyword and value can be separated by only white space:
\begin{verbatim}
keyword            value
\end{verbatim}
or the keyword and value can be separated by an equal sign and white space:
\begin{verbatim}
keyword      =     value
\end{verbatim}
Blank lines in the configuration file are ignored.  Comments are prefaced by
a \verb!#! and may appear on the end of a line with actual values:
\begin{verbatim}
keyword            value          #  This is a comment
\end{verbatim}
or may be at the beginning of a line:
\begin{verbatim}
#  This entire line is a comment . . . 
\end{verbatim}
Some keywords require several lines of data.
These are generally implemented to either allow the data to be read from a file:
\begin{verbatim}
keyword            filename
\end{verbatim}
or to be included inline using Tcl-style braces:
\begin{verbatim}
keyword {
  lots of data
}
\end{verbatim}

The specification of the keywords is case insensitive 
so that any combination of 
upper and lower case letters will have the same meaning.  
Hence, {\tt DCDfile} and {\tt dcdfile} 
are equivalent.  The capitalization in the values, however, may be important.
Some values indicate file names, in which capitalization is critical.  
Other values such as {\tt on} or {\tt off} are case insensitive.

\subsubsection{Tcl scripting interface and features}
\label{section:tclscripting}

When compiled with Tcl (all released binaries) the config file
is parsed by Tcl in a fully backwards compatible manner with the
added bonus that any Tcl command may also be used.  This alone allows:
\begin{itemize}
 \item the "source" command to include other files (works w/o Tcl too!),
 \item the "print" command to display messages ("puts" is broken, sorry),
 \item environment variables through the env array ("\$env(USER)"), and
 \item user-defined variables ("set base sim23", "dcdfile \${base}.dcd").
\end{itemize}

Additional features include:
\begin{itemize}
 \item The "callback" command takes a 2-parameter Tcl procedure which
    is then called with a list of labels and a list of values during
    every timestep, allowing analysis, formatting, whatever.
 \item The "run" command takes a number of steps to run (overriding the
    now optional numsteps parameter, which defaults to 0) and can be
    called repeatedly.  You can "run 0" just to get energies.
 \item The "minimize" command is similar to "run" and performs
    minimization for the specified number of force evaluations.
 \item The "output" command takes an output file basename and causes
    .coor, .vel, and .xsc files to be written with that name.
 \item Between "run" commands the reassignTemp, rescaleTemp, and
    langevinTemp parameters can be changed to allow simulated
    annealing protocols within a single config file.  (Many more
    parameters of this type will be enabled in future versions.)
 \item The "checkpoint" and "revert" commands (no arguments) allow
    a scripted simulation to save and restore to a prior state.
    (The minimizer makes use of these commands also - you cannot
    revert to a checkpoint made before the last minimize call.)
 \item The "reinitvels" command reinitializes velocities to a
    random distribution based on the given temperature.
 \item The "measure" command allows user-programmed calculations to
    be executed in order to facilitate automated methods.  (For
    example, to revert or change a parameter.)  You will need to
    write code and compile NAMD to make use of this feature.
\end{itemize}

Please note that while NAMD has traditionally allowed comments to be
started by a \# appearing anywhere on a line, Tcl only allows comments
to appear where a new statement could begin.  With Tcl config file
parsing enabled (all shipped binaries) both NAMD and Tcl comments are
allowed before the first "run" command.  At this point only pure Tcl
syntax is allowed.  In addition, the ";\#" idiom for Tcl comments will
only work with Tcl enabled.  NAMD has also traditionally allowed
parameters to be specified as "param=value".  This is supported, but
only before the first "run" command.  Some examples:

\begin{verbatim}
# this is my config file                            <- OK
reassignFreq 100 ; # how often to reset velocities  <- only w/ Tcl
reassignTemp 20 # temp to reset velocities to       <- OK before "run"
run 1000                                            <- now Tcl only
reassignTemp 40 ; # temp to reset velocities to     <- ";" is required
\end{verbatim}

NAMD has also traditionally allowed parameters to be specified as
"param=value" as well as "param value".  This is supported, but only
before the first "run" command.  For an easy life, use "param value".

\subsubsection{Required \NAMD\ configuration parameters}
\label{section:requiredparams}

The following parameters are {\em required} for every
\NAMD\ simulation:

\begin{itemize}

\item
{\tt numsteps} (page \pageref{param:numsteps}),

\item
{\tt coordinates} (page \pageref{param:coordinates}),

\item
{\tt structure} (page \pageref{param:structure}),

\item
{\tt parameters} (page \pageref{param:parameters}),

\item
{\tt exclude} (page \pageref{param:exclude}), 

\item
{\tt outputname} (page \pageref{param:outputname}), 

\item
one of the following three:
\begin{itemize}
\item
{\tt temperature} (page \pageref{param:temperature}),

\item
{\tt velocities} (page \pageref{param:velocities}),

\item
{\tt binvelocities} (page \pageref{param:binvelocities}).
\end{itemize}

\end{itemize}

\noindent These required parameters specify the most basic properties of
the simulation.  %  that is to be performed.
In addition, it is highly recommended that 
{\tt pairlistdist} be specified with a 
value at least one greater than {\tt cutoff}.

