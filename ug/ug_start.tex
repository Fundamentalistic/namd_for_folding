
\section{Getting Started}
\label{section:start}

\subsection{What is needed}

Before running \NAMD, explained in section \ref{section:run}, 
the following are be needed:
\begin{itemize}
\item A CHARMM force field in either CHARMM or X-PLOR format.
\item
An X-PLOR format PSF file describing the modelular structure.
\item
The initial coordinates of the molecular system in the form of a PDB file.  
\item
A \NAMD\ configuration file.
\end{itemize}

We strongly recommend that you have access to either CHARMM or X-PLOR,
either of which is capable of generating both the PSF and PDB files.
\NAMD\ currently provides no automatic method of generating these files.

\subsection{\NAMD\ configuration file}
\label{section:config}

Besides these input and output files, \NAMD\ also uses 
a file referred to as the {\it configuration file\/}.  
This file specifies what dynamics options and values that 
\NAMD\ should use, such as the number of timesteps to perform, 
initial temperature, etc.  
The options and values in this file control how 
the system will be simulated.  

A \NAMD\ configuration file contains a set of options and values.  
The options and values specified determine the exact behavior of
\NAMD, what features are active or inactive, how long the simulation
should continue, etc.  Section \ref{section:configsyntax} describes how
options are specified within a \NAMD\ configuration file.  Section
\ref{section:requiredparams} lists the parameters which are required
to run a basic simulation.  Section \ref{section:xplorequiv}
describes the relation between specific \NAMD\ and X-PLOR dynamics
options.  Several sample \NAMD\ configuration files are shown
in section \ref{section:sample}.


\subsubsection{Configuration parameter syntax}
\label{section:configsyntax}
Each line
in the configuration files consists of a $keyword$ identifying the option
being specified, and a $value$ which is a parameter to be used for this
option.  The keyword and value can be separated by only white space:
\begin{verbatim}
keyword            value
\end{verbatim}
or the keyword and value can be separated by an equal sign and white space:
\begin{verbatim}
keyword      =     value
\end{verbatim}
Blank lines in the configuration file are ignored.  Comments are prefaced by
a \verb!#! and may appear on the end of a line with actual values:
\begin{verbatim}
keyword            value          #  This is a comment
\end{verbatim}
or may be at the beginning of a line:
\begin{verbatim}
#  This entire line is a comment . . . 
\end{verbatim}
Some keywords require several lines of data.
These are generally implemented to either allow the data to be read from a file:
\begin{verbatim}
keyword            filename
\end{verbatim}
or to be included inline using Tcl-style braces:
\begin{verbatim}
keyword {
  lots of data
}
\end{verbatim}

The specification of the keywords is case insensitive 
so that any combination of 
upper and lower case letters will have the same meaning.  
Hence, {\tt DCDfile} and {\tt dcdfile} 
are equivalent.  The capitalization in the values, however, may be important.
Some values indicate file names, in which capitalization is critical.  
Other values such as {\tt on} or {\tt off} are case insensitive.

\subsubsection{Required \NAMD\ configuration parameters}
\label{section:requiredparams}

The following parameters are {\em required} for every
\NAMD\ simulation:

\begin{itemize}

\item
{\tt numsteps} (page \pageref{param:numsteps}),

\item
{\tt coordinates} (page \pageref{param:coordinates}),

\item
{\tt structure} (page \pageref{param:structure}),

\item
{\tt parameters} (page \pageref{param:parameters}),

\item
{\tt exclude} (page \pageref{param:exclude}), 

\item
{\tt outputname} (page \pageref{param:outputname}), 

\item
one of the following three:
\begin{itemize}
\item
{\tt temperature} (page \pageref{param:temperature}),

\item
{\tt velocities} (page \pageref{param:velocities}),

\item
{\tt binvelocities} (page \pageref{param:binvelocities}).
\end{itemize}

\end{itemize}

\noindent These required parameters specify the most basic properties of
the simulation.  %  that is to be performed.
In addition, it is highly recommended that 
{\tt pairlistdist} be specified with a 
value at least one greater than {\tt cutoff}.

