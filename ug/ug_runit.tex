%%%%%%%%%%%%%%%%%%%%%%%%%%%%%%%%%%%%%%%%%%%%%%%%%%%%%%%%%%%%%%%%%%%%%%%%%%%%
%                                                                          %
%              (C) Copyright 1995 The Board of Trustees of the             %
%                          University of Illinois                          %
%                           All Rights Reserved                            %
%								  	   %
%%%%%%%%%%%%%%%%%%%%%%%%%%%%%%%%%%%%%%%%%%%%%%%%%%%%%%%%%%%%%%%%%%%%%%%%%%%%

%%%%%%%%%%%%%%%%%%%%%%%%%%%%%%%%%%%%%%%%%%%%%%%%%%%%%%%%%%%%%%%%%%%%%%%%%%%%
% RCS INFORMATION:
%
%       $RCSfile: ug_runit.tex,v $
%       $Author: dhardy $        $Locker:  $                $State: Exp $
%       $Revision: 1.1 $      $Date: 1998/01/05 21:12:34 $
%
%%%%%%%%%%%%%%%%%%%%%%%%%%%%%%%%%%%%%%%%%%%%%%%%%%%%%%%%%%%%%%%%%%%%%%%%%%%%
% DESCRIPTION:
%	This file contains the section of the NAMD User's Guide which
% describes how to actually run the program.
%
%%%%%%%%%%%%%%%%%%%%%%%%%%%%%%%%%%%%%%%%%%%%%%%%%%%%%%%%%%%%%%%%%%%%%%%%%%%%
% REVISION HISTORY:
%
% $Log: ug_runit.tex,v $
% Revision 1.1  1998/01/05 21:12:34  dhardy
% user guide, first draft
%
% Revision 1.8  1997/08/07 17:28:20  dhardy
% minor revisions
%
% Revision 1.7  1997/08/07 15:54:32  dhardy
% edit presentation style
%
% Revision 1.6  1997/06/23 18:29:39  nealk
% Updated setenv stuff.
%
% Revision 1.5  1997/04/24 15:10:11  nealk
% Added items for machine-specific environment settings (setenv).
%
% Revision 1.4  1996/05/15 19:21:07  jean
% ready (I hope) for 1.4 beta release
%
% Revision 1.3  1995/06/30 15:51:43  brunner
% More complete.  Now includes sections for everything except Exemplar
% and Charm++.  Still needs proofreading.
%
% Revision 1.2  1995/06/29  16:39:04  brunner
% Single node and PVM instructions complete.  The rest still need work.
%
% Revision 1.1  95/06/19  16:08:17  16:08:17  nelson (Mark T. Nelson)
% Initial revision
% 
%%%%%%%%%%%%%%%%%%%%%%%%%%%%%%%%%%%%%%%%%%%%%%%%%%%%%%%%%%%%%%%%%%%%%%%%%%%%

\section{Running \NAMD}
\label{section:run}

\subsection{Single node execution}

To run \NAMD\ on a single node, type {\tt namd config-file}, where
{\tt config-file} is the configuration file described in Section
\ref{section:config}.

\subsection{Parallel execution with PVM}

Parallel execution using PVM requires that PVM version 3.x be
installed on each machine you intend to use.  Consult the PVM
documentation and your system administrator to determine if this is
the case.
\prettypar
The command to run \NAMD\ is the same for the PVM version as for
single-node execution.  One of the first things \NAMD\ does is call a
PVM function to determine how many nodes are available.  If this
function fails, the program knows it is running on a single machine,
and runs in single-processor mode.  Otherwise it runs in parallel
mode, using all available nodes, except for the FMA calculations,
which use only the greatest power of two number of processors not
greater than the total number of processors.

%%%%%%%%%%%%%%%%%%%%%%%%%%%%%%%%%%%%%%%%%%%%%%%%%%%%%%%%%%%%
%  Maybe we need to leave this out for the time being.  RKB
%
%\subsection{Parallel execution with Charm++}
%
%Need Charm++
%
%Running varies according to machine.  See Charm++ docs for info.
%
%%%%%%%%%%%%%%%%%%%%%%%%%%%%%%%%%%%%%%%%%%%%%%%%%%%%%%%%%%%%

\subsection{Platform specific notes}

\subsubsection{Cluster of workstations}

Before using \NAMD\ with PVM for the first time, you must:
\begin{itemize}
  \item Make sure that {\tt pvmd} is running on each machine you plan
to use.

  \item Have a directory called {\tt pvm3} in your home
directory on each machine you plan to use.  If not, consult the PVM
documentation for information on creating this directory.

  \item Create a symbolic link to the {\tt namd} executable file in
every {\tt pvm3/bin/MACHINE-TYPE} directory corresponding to the
machine types you plan to use.  You will also need links to {\tt
dpmta-slave} if you wish to use FMA.

  \item Set any required environment variables.
The {\tt PVM\_ARCH} environment variable defines the type of operating system.
The {\tt PVM\_ROOT} environment variable is the location of the PVM directory.
We will assume the PVM root directory is {\tt /home/pvm3}.
The {\tt PVMBUFSIZE} environment variable overrides the default PVM buffer size.
  \begin{itemize}
    {
    \item for HP 735's: \\
      {\tt setenv PVM\_ARCH HPPA} \\
      {\tt setenv PVM\_ROOT /home/pvm3}

    \item for HP K-machines: \\
      {\tt setenv PVM\_ARCH HPPAMP} \\
      {\tt setenv PVM\_ROOT /home/pvm3} \\
      {\tt setenv PVMBUFSIZE 16777216} \\
      Note: PVMBUFSIZE is a variable which determines how big messages
      can be on the HPPAMP machines.  On a 36,000 atom system, 1MB (the
      default) was too small, so we suggest 16MB.  If your system fails
      shortly after NAMD starts, you might try increasing the value.
    }
  \end{itemize}

\end{itemize}

To run \NAMD, you must first use the {\tt pvm} command to select which
machines to use.  Then run {\tt namd config-file} as you would for a
single-machine run.  When \NAMD\ is done, run {\tt pvm} again to clean
up.  The following is an example of how to run \NAMD\ on four
processors.
\begin{verbatim}
caliban(233) pvm
pvm> add ophelia
1 successful
                    HOST     DTID
                 ophelia    80000
pvm> add romeo
1 successful
                    HOST     DTID
                   romeo    c0000
pvm> add juliet
1 successful
                    HOST     DTID
                  juliet   100000
pvm> conf
4 hosts, 1 data format
                    HOST     DTID     ARCH   SPEED
                 caliban    40000     HPPA    1000
                 ophelia    80000     HPPA    1000
                   romeo    c0000     HPPA    1000
                  juliet   100000     HPPA    1000
pvm> quit

pvmd still running.
\end{verbatim}

The user is logged in to caliban.  He runs {\tt pvm}, and uses the
{\tt add machine} to select machines ophelia, romeo, and juliet in
addition to caliban.  The command {\tt conf} shows what machines have
been selected.

\begin{verbatim}
caliban(234) namd alanin
[...namd output deleted...]
Node 0:Info> TOTAL RUN TIME 24.0511
Node 0:Info> TOTAL CPU TIME 11.78
Node 0:Info> namd exiting
caliban(235) pvm
pvmd already running.
pvm> halt
caliban(236) 
\end{verbatim}

The user runs \NAMD\ using the configuration file {\tt alanin}.  When
the simulation is complete, the user runs {\tt pvm} again, and issues
the {\tt halt} command to tell PVM to free the four processors for
another run.

\subsubsection{Cray T3D}

The Cray T3D implementation uses the T3D PVM library, so program
execution is similar to running in the workstation cluster
environment.  The main difference is that the number of processors is
set up through environment variables and the T3D process queuing
system, so the steps described above involving the {\tt pvm} control
program are not necessary.  Instead, the user must set up certain
variables as described in the Cray T3D user's manual.

%\subsubsection{Convex Exemplar}

\subsection{Interactive modeling with MDScope}

Interactive molecular modeling can be performed using the MDScope
environment.  MDScope consists of \NAMD; \VMD, a program for
interactive visualization of biopolymers; and MDComm, a communications
library for efficient network transfer of molecular dynamics
information.  \VMD\ allows the user to view intermediate results of the
simulation being performed by \NAMD\ while the simulation is still 
proceeding.  For more information regarding 
the installation or use of MDScope, 
consult the \VMD\ documentation.  

