%%%%%%%%%%%%%%%%%%%%%%%%%%%%%%%%%%%%%%%%%%%%%%%%%%%%%%%%%%%%%%%%%%%%%%%%%%%%
%                                                                          %
%              (C) Copyright 1995 The Board of Trustees of the             %
%                          University of Illinois                          %
%                           All Rights Reserved                            %
%								  	   %
%%%%%%%%%%%%%%%%%%%%%%%%%%%%%%%%%%%%%%%%%%%%%%%%%%%%%%%%%%%%%%%%%%%%%%%%%%%%

%%%%%%%%%%%%%%%%%%%%%%%%%%%%%%%%%%%%%%%%%%%%%%%%%%%%%%%%%%%%%%%%%%%%%%%%%%%%
% RCS INFORMATION:
%
%       $RCSfile: namd_macros.tex,v $
%       $Author: jim $        $Locker:  $                $State: Exp $
%       $Revision: 1.3 $      $Date: 1999/03/18 01:49:49 $
%
%%%%%%%%%%%%%%%%%%%%%%%%%%%%%%%%%%%%%%%%%%%%%%%%%%%%%%%%%%%%%%%%%%%%%%%%%%%%
% DESCRIPTION:
%
%%%%%%%%%%%%%%%%%%%%%%%%%%%%%%%%%%%%%%%%%%%%%%%%%%%%%%%%%%%%%%%%%%%%%%%%%%%%
% REVISION HISTORY:
%
% $Log: namd_macros.tex,v $
% Revision 1.3  1999/03/18 01:49:49  jim
% Many changes, now mostly ready for 2.0 release.
%
% Revision 1.2  1999/02/26 06:57:15  jim
% Changes to make latex2html work.
%
% Revision 1.1  1998/01/05 21:12:26  dhardy
% user guide, first draft
%
% Revision 1.2  1995/06/23 14:35:27  nelson
% Change mycite macro to actual cite articles.
%
% Revision 1.1  95/06/19  16:09:27  16:09:27  nelson (Mark T. Nelson)
% Initial revision
% 
%%%%%%%%%%%%%%%%%%%%%%%%%%%%%%%%%%%%%%%%%%%%%%%%%%%%%%%%%%%%%%%%%%%%%%%%%%%%

%
% generally useful macros
%
\newcommand{\REFAND} {\&}
\newcommand{\ETALNP}{\mbox{\it et al}}
\newcommand{\ETAL}{\mbox{\ETALNP{\it.}}}
\newcommand{\eqnref}[1] {\mbox{eq (\ref{#1})}}
\newcommand{\mycite}[2] {\cite{#2}}
%\newcommand{\mycite}[2] {}

%
% include name and version number of program; this is generated
% by the 'make version' command in either the doc or src directory
% This will define the 'NAMDNAME', 'NAMDVER', and 'NAMDDATE' macros,
% as well as the 'NAMDAUTHORS' macro.
% Then the 'NAMD' macro is used throughout the document to refer to
% the program name, while the other macros as defined in the file
% namd_version.tex are used as given.
%


\newcommand{\NAMD} {NAMD}
\newcommand{\NAMDDATE} {\today}
\newcommand{\NAMDVER} {2.3b1}
\newcommand{\NAMDAUTHORS} {M.~Bhandarkar, R.~Brunner, A.~Dalke, J.~Gullingsrud, A.~Gursoy, W.~Humphrey, D.~Hurwitz, N.~Krawetz, M.~Nelson, J.~Phillips, A.~Shinozaki}



%
% macros for style conventions when describing the program.
%

% name of class or object in program
\newcommand{\OBJ}[1] {{\bf\tt#1}}

% function arguments
\newcommand{\FA}[2] {{\rm{\bf#1}\ {\it#2}}}

% global function name
\newcommand{\FN}[3] {{\rm\bf#1}\ {\tt #2(}#3{\tt)}}

% class member function name
\newcommand{\FNO}[4] {{\rm\bf#2}\ \OBJ{#1::}{\tt#3(}#4{\tt)}}

% list item, for optional components, parameters, etc.
\newcommand{\TTLISTITEM}[1] {\item {\tt #1} \\}
\newcommand{\RMLISTITEM}[1] {\item {\rm #1} \\}
\newcommand{\BOLDLISTITEM}[1] {\item {\bf #1} \\}
\newcommand{\EMLISTITEM}[1] {\item {\em #1} \\}
\newcommand{\LISTITEM}[1] {\RMLISTITEM{#1}}

%
% other generally useful macros
%

% Other program names, formatted nicely
\newcommand{\VMD} {VMD}
\newcommand{\MDCOMM} {MDComm}
\newcommand{\MDSCOPE} {MDScope}
\newcommand{\CESB} {MDScope}
\newcommand{\ALLNAMES} {MDScope}
\newcommand{\SMALLMDSCOPE} {mdscope}
\newcommand{\SMALLCESB} {mdscope}
\newcommand{\SMALLALLNAMES} {mdscope}

% full name for MDScope, i.e., what it stands for
\newcommand{\MDSCOPENAME} {Molecular Dynamics computational environment}

% title of MDScope paper
\newcommand{\MDSCOPEPAPER} {MDScope: A Visual Computing Environment for
Structural Biology}

% default definitions for title and description
% \newcommand{\DOCTITLE} {Documentation Guides}
% \newcommand{\DOCDESC} {
%   This document includes the \NAMD\ Installation, Users, and
% Programmers Guides, which document how to obtain, install, use, and
% modify the molecular graphics program \NAMD.}

