
\section{Accelerated Sampling Methods}
\label{section:accel}


\subsection{Locally enhanced sampling}
\label{section:les}

Locally enhanced sampling (LES)~\cite{ROIT91,SIMM98,SIMM00} increases
sampling and transition rates for a portion of a molecule by the use of
multiple non-interacting copies of the enhanced atoms.  These enhanced
atoms experience an interaction (electrostatics, van der Waals, and
covalent) potential that is divided by the number of copies present.
In this way the enhanced atoms can occupy the same space, while the
multiple instances and reduces barriers increase transition rates.

\subsubsection{Structure generation}

To use LES, the structure and coordinate input files must be modified to
contain multiple copies of the enhanced atoms.  \PSFGEN\ provides the
{\tt multiply} command for this purpose.  \NAMD\ supports a maximum of 15
copies, which should be sufficient.  

Begin by generating the complete molecular structure and guessing
coordinates as described in Sec.~\ref{section:psfgen}.  As the last
operation in your script, prior to writing the psf and pdb files, add
the {\tt multiply} command, specifying the number of copies desired and
listing segments, residues, or atoms to be multiplied.  For example,
\verb#multiply 4 BPTI:56 BPTI:57# will create four copies of the last
two residues of segment BPTI.  You must include all atoms to be
enhanced in a single {\tt multiply} command in order for the bonded
terms in the psf file to be duplicated correctly.  Calling {\tt multiply}
on connected sets of atoms multiple times will produce unpredictable
results, as may running other commands after {\tt multiply}.

The enhanced atoms are duplicated exactly in the structure---they have
the same segment, residue, and atom names.  They are distinguished only
by the value of the B (beta) column in the pdb file, which is 0 for
normal atoms and varies from 1 to the number of copies created for
enhanced atoms.  The enhanced atoms may be easily observed in VMD with
the atom selection \verb#beta != 0#.

\subsubsection{Simulation}

In practice, LES is a simple method used to increase sampling;
no special output is generated.
The following parameters are used to enable LES:

\begin{itemize}

\item
\NAMDCONFWDEF{les}{is locally enhanced sampling active?}{{\tt on} or {\tt
off}}{{\tt off}}
{Specifies whether or not LES is active.}

\NAMDCONF{lesFactor}{number of LES images to use}
{positive integer equal to the number of images present}
{This should be equal to the factor used in {\tt multiply}
 when creating the structure.  The interaction potentials for images is
 divided by {\tt lesFactor}.  
}

\item
\NAMDCONFWDEF{lesReduceTemp}{reduce enhanced atom temperature?}{{\tt on} or {\tt
off}}{{\tt off}}
{Enhanced atoms experience interaction potentials divided by {\tt lesFactor}.
This allows them to enter regions that would not normally be thermally
accessible.  If this is not desired, then the temperature of these atoms
may be reduced to correspond with the reduced potential.  This option
affects velocity initialization, reinititialization, reassignment, and
the target temperature for langevin dynamics.  Langevin dynamics is
recommended with this option, since in a constant energy simulation energy
will flow into the enhanced degrees of freedom until they reach thermal
equilibrium with the rest of the system.  The reduced temperature atoms
will have reduced velocities as well, unless {\tt lesReduceMass} is also
enabled.}

\item
\NAMDCONFWDEF{lesReduceMass}{reduce enhanced atom mass?}{{\tt on} or {\tt off}}{{\tt off}}
{Used with {\tt lesReduceTemp} to restore velocity distribution to
enhanced atoms.  If used alone, enhanced atoms would move faster than
normal atoms, and hence a smaller timestep would be required.}

\item
\item
\NAMDCONFWDEF{lesFile}{PDB file containing LES flags}{UNIX filename} {{\tt coordinates}}
{PDB file to specify the LES image number of each atom.
If this parameter is not specified, then 
the PDB file containing initial coordinates specified by 
{\tt coordinates} is used.}

\item
\NAMDCONFWDEF{lesCol}{column of PDB file containing LES flags}{{\tt X}, {\tt Y}, {\tt Z}, {\tt O}, or {\tt B}}{{\tt B}}
{Column of the PDB file to specify the LES image number of each atom.
This parameter may specify any of the floating point fields of the PDB file, 
either X, Y, Z, occupancy, or beta-coupling (temperature-coupling).  
A value of 0 in this column indicates that the atom is not enhanced.
Any other value should be a positive integer less than {\tt lesFactor}.}

\end{itemize}


\subsection{Replica exchange simulations}

\index{replica exchange}
The {\tt lib/replica/}
directory contains Tcl scripts that implement replica exchange
for NAMD, using a Tcl server and socket connections to drive a
separate NAMD process for every replica used in the simulation.
Replica exchanges and energies are recorded in the potenergy.dat,
realtemp.dat, and targtemp.dat files written in the output directory.
These can be viewed with, e.g., ``{\tt xmgrace -nxy ....potenergy.dat}''
There is also a script to load the output into VMD and color each
frame according to target temperature.  An example simulation folds
a 66-atom model of a deca-alanine helix in about 10\,ns.

This implementation is designed to be modified by the user to implement
exchanges of parameters other than temperature or via other temperature
exchange methods.  The scripts should provide a good starting point for
any simulation method requiring a number of loosely interacting systems.

{\tt replica\_exchange.tcl}
is the master Tcl script for replica exchange simulations, it is run in
{\tt tclsh} {\em outside of NAMD} and takes a replica exchange config
file as an argument:
\begin{verbatim}
          tclsh ../replica_exchange.tcl fold_alanin.conf
          tclsh ../replica_exchange.tcl restart_1.conf
\end{verbatim}
{\tt replica\_exchange.tcl} uses code in
{\tt namd\_replica\_server.tcl}, a general script for driving NAMD slaves, and
{\tt spawn\_namd.tcl}, a variety of methods for launching NAMD slaves.

{\tt show\_replicas.vmd} is a script for loading replicas into VMD;
first source the replica exchange conf file and then this script, then
repeat for each restart conf file or for example just do
``{\tt vmd -e load\_all.vmd}''.
This script will likely destroy anything else you are doing in VMD at the
time, so it is best to start with a fresh VMD.
{\tt clone\_reps.vmd} provides the {\tt clone\_reps} commmand to copy graphical
representation from the top molecule to all other molecules.

A replica exchange config file should define the following Tcl variables:
\begin{itemize}
\item {\tt num\_replicas}, the number of replica simulations to use,
\item {\tt min\_temp}, the lowest replica target temperature,
\item {\tt max\_temp}, the highest replica target temperature,
\item {\tt steps\_per\_run}, the number of steps between exchange attempts,
\item {\tt num\_runs}, the number of runs before stopping
(should be divisible by {\tt runs\_per\_frame} $\times$ {\tt frames\_per\_restart}).
\item {\tt runs\_per\_frame}, the number of runs between trajectory outputs,
\item {\tt frames\_per\_restart}, the number of frames between restart outputs,

\item {\tt namd\_config\_file}, the NAMD config file containing all parameters,
needed for the simulation except {\tt seed}, {\tt langevin}, 
{\tt langevinDamping}, {\tt langevinTemp}, {\tt outputEnergies},
{\tt outputname}, {\tt dcdFreq},
{\tt temperature}, {\tt bincoordinates}, {\tt binvelocities},
or {\tt extendedSystem}, which are provided by {\tt replica\_exchange.tcl},

\item {\tt output\_root}, the directory/fileroot for output files,

\item {\tt psf\_file}, the psf file for {\tt show\_replicas.vmd}, 
\item {\tt initial\_pdb\_file}, the initial coordinate pdb file for {\tt show\_replicas.vmd},
\item {\tt fit\_pdb\_file}, the coodinates that frames are fit to by {\tt show\_replicas.vmd} (e.g., a folded structure),
\item {\tt server\_port}, the port to connect to the replica server on, and
\item {\tt spawn\_namd\_command}, a command from {\tt spawn\_namd.tcl} and arguments to launch NAMD jobs.
\end{itemize}

The {\tt lib/replica/example/} directory contains
all files needed to fold a 66-atom model of a deca-alanine helix:
\begin{itemize}
\item {\tt alanin\_base.namd}, basic config options for NAMD,
\item {\tt alanin.params}, parameters,
\item {\tt alanin.psf}, structure,
\item {\tt unfolded.pdb}, initial coordinates,
\item {\tt alanin.pdb}, folded structure for fitting in {\tt show\_replicas.vmd},
\item {\tt fold\_alanin.conf}, config file for {\tt replica\_exchange.tcl} script,
\item {\tt restart\_1.conf}, config file to continue alanin folding another 10\,ns, and
\item {\tt load\_all.vmd}, load all output into VMD and color by target temperature.
\end{itemize}

The {\tt fold\_alanin.conf} config file contains the following settings:
\begin{verbatim}
set num_replicas 8
set min_temp 300
set max_temp 600
set steps_per_run 1000
set num_runs 10000
set runs_per_frame 10
set frames_per_restart 10
set namd_config_file "alanin_base.namd"
set output_root "output/fold_alanin" ; # directory must exist
set psf_file "alanin.psf"
set initial_pdb_file "unfolded.pdb"
set fit_pdb_file "alanin.pdb"
set namd_bin_dir /Projects/namd2/bin/current/Linux64
set server_port 3177
set spawn_namd_command \
  [list spawn_namd_ssh "cd [pwd]; [file join $namd_bin_dir namd2] +netpoll" \
  [list beirut belfast] ]
\end{verbatim}

\subsection{Random acceleration molecular dynamics simulations}

The "lib/ramd" directory stores the tcl scripts and the example files for the implementation of the Random Acceleration Molecular Dynamics (RAMD) simulation method in NAMD. 
The RAMD method can be used to carry out molecular dynamics (MD) simulations with an additional randomly oriented acceleration applied to the center of mass of one group of atoms (referred below as "ligand") in the system. 
It can, for example, be used to identify egress routes for a ligand from a buried protein binding site. 
Since its original implementation in the ARGOS (ref 1, 2) program, the method has been also implemented in AMBER 8 (ref 3), and CHARMM (ref 4). 
The first implementation of RAMD in NAMD using a tcl script (available as supplementary material in ref 6) provided only limited functionality compared to the AMBER 8 implementation. 

In the current implementation, the RAMD method can be performed in 2 flavors: (i) "pure RAMD simulations" in which the randomly-oriented acceleration is applied continuously, and (ii) "combined RAMD-MD simulations" in which RAMD steps alternate with standard MD steps. 
Additional information is found in the README file in the "lib/ramd" directory. 
The user is encouraged to carefully read this information before starting production runs.

The three required scripts are stored in "lib/ramd/scripts": (i) ramd--4.0.tcl defines the simulation parameters and passes them from the NAMD configuration file to the main script, (ii) "ramd--4.0\_script.tcl" adds the randomly oriented force and performs all related computations, and  (iii) "vectors.tcl" was borrowed from VMD and defines the vector operations used.

Two examples for running the scripts are included in the directory "lib/ramd/examples".  
The user is encouraged to read the "README.examples" file provided in the same directory.

In order to turn RAMD on, the line "source /path/to/your/files/ramd--4.0.tcl" should be included in the NAMD configuration file. 
Unless the user decides to store the scripts at a different location, the path 
"/path/to/your/files" should point to the "lib/ramd/scripts" directory. 
Otherwise, the user should make sure that the directory "/path/to/your/files" stores all three scripts described above. 

The specific RAMD simulation parameters to be provided in the NAMD configuration file (listed below) should be preceded by the keyword "ramd". 
The default values for these parameters are only given as guidance. 
They are likely not to be suitable for other systems than those the scripts were tested on. 

\begin{itemize}

\item
\NAMDCONFWDEF{ramd debugLevel}{ Set debug level of RAMD} {\tt integer value } {0} { Activates verbose output if set to an integer greater than 0. Should be used only for testing purposes because the very dense output is full of information only relevant for debugging.}

\NAMDCONFWDEF{ramd mdStart}{ Start RAMD-MD with MD or RAMD?} {{\tt yes} or {\tt no}} {{\tt no}} { Specifies if combined RAMD-MD simulation starts with MD or RAMD steps; ignored if pure RAMD simulation is performed.  Should be set to "yes" if initial MD steps are desired.}

\item
\NAMDCONFWDEF{ramd ramdSteps} { Set number steps in RAMD block} {{\tt positive integer}} {{\tt 50}} {Specifies the number of steps in 1 RAMD block; the simulations are evaluated every 'ramdSteps' steps.} 
 
\item
\NAMDCONFWDEF{ramd mdSteps} { Set number steps in MD block} {{\tt positive integer}} {{\tt 0}} {Specifies the number of steps in 1 standard MD block; in combined RAMD-MD simulations, the RAMD blocks are evaluated every 'ramdSteps', the MD blocks every 'mdSteps' steps. Default of 0 gives pure RAMD simulation.}

\item
\NAMDCONFWDEF{ramd accel} {Set acceleration energy} {{\tt positive decimal}} {{\tt 0.25}} {Specifies acceleration in kcal/mol*A*amu to be applied during RAMD step.}

\item
\NAMDCONFWDEF{ramd rMinRamd} {Set threshold for distance traveled RAMD} {{\tt positive decimal}} {{\tt 0.01}} {Specifies a threshold value for the distance in Angstroms traveled by the ligand in 1 RAMD block. In pure RAMD simulations the direction of the acceleration is changed if the ligand traveled less than 'rMinRamd' \AA in the evaluated block. In combined RAMD-MD simulations, a switch from a RAMD block to a standard MD block is applied if the ligand traveled more than 'rMinRamd' \AA in the evaluated block.}

\item
\NAMDCONF{ramd rMinMd} {Set threshold for distance traveled in MD} {{\tt positive decimal}} {Specifies a threshold value for the distance, in Angstroms, traveled by accelerated atoms in 1 standard MD block.  In combined RAMD-MD simulations, a switch from a standard MD block to a RAMD block is applied according to the criteria described in the note below.  Required if 'mdStep' is not 0; ignored if 'mdSteps' is 0.}
 
\item
\NAMDCONFWDEF{ramd forceOutFreq}{Set frequency of RAMD forces output} {{\tt positive integer}, Must be divisor of both {\tt ramdSteps} and {\tt mdSteps}} {{\tt 0}} { Every 'forceOutFreq' steps, detailed output of forces will be written.} 

\item
\NAMDCONFWDEF{ramd maxDist} {Set center of mass separation} {{\tt positive decimal}} {{\tt 50}} { Specifies the distance in Angstroms between the the centers of mass of the ligand and the protein when the simulation is stopped.}
 
\item
\NAMDCONFWDEF{ramd firstProtAtom} {First index of protein atom} {{\tt positive integer}} {{\tt 1}} { Specifies the index of the first protein atom.}
 
\item
\NAMDCONF{ramd lastProtAtom} {Last index of protein atom} {{\tt positive atom}} { Specifies the index of the last protein atom. } 

\item
\NAMDCONF{ramd firstRamdAtom}{ First index of ligand atom } {{\tt positive integer}} {Specifies the index of the first ligand atom.}

\item
\NAMDCONF{ramd lastRamdAtom}{ Last index of ligand atom } {{\tt positive integer}} {Specifies the index of the last ligand atom. }

\item
\NAMDCONFWDEF{ramd ramdSeed}{Set RAMD seed} {{\tt positive integer}} {{\tt 14253}} {Specifies seed for the random number generator for generation of acceleration directions. Change this parameter if you wish to run different trajectories with identical parameters.}

\end{itemize}

Note: 
In combined RAMD-MD simulations, RAMD blocks alternate with standard MD blocks ('ramdSteps' and 'mdSteps' input parameters). The switches between RAMD and MD blocks are decided based on the following parameters: (i) 'd' =  the distance between the protein and ligand centers of mass, (ii) 'dr' = the distance traveled by the ligand in 1 RAMD block, and (iii) 'dm' = the distance traveled by the ligand in 1 MD block. A switch from RAMD to MD is applied if 'dr' > 'rRamdMin'. A switch from MD to RAMD is applied if: (i) 'dm' < 'rMdMin' and 'd' > 0 (acceleration direction is kept from previous RAMD block), (ii) if 'dm' < 'rMdMin' and 'd' < 0 (acceleration direction is changed), (iii) if 'dm' > 'rMdMin' and 'd' < 0 (acceleration direction is changed). In all other case, a switch is not applied.
 
%References:
%Luedemann, S.K., Lounnas, V. and R. C. Wade. How do Substrates Enter and Products Exit the Buried Active Site of Cytochrome P450cam ? 1. Random Expulsion Molecular Dynamics Investigation of Ligand Access Channels and Mechanisms. J Mol Biol, 303:797-811 (2000). 
%Winn,P., Luedemann, S.K., Gauges,R., Lounnas, V. and R. C. Wade. Comparison of the dynamics of substrate access channels in three cytochrome P450s reveals different opening mechanisms and a new functional role for a buried arginine PNAS, 99, 5361-5366 (2002). 
%Schleinkofer, K., Sudarko, Winn,P., Luedemann, S.K. and R. C. Wade. Do mammalian cytochrome P450s show multiple ligand access pathways and ligand channelling? EMBO Reports, 6, 584-589 (2005).
%Carlsson, P., Burendahl, S., Nilsson, L. Unbinding of retinoic acid from the retinoic acid receptor by random expulsion molecular dynamics. Biophys. J. 91, 3151-3161 (2006).
%Wang, T., Duan, Y. Chromophore channeling in the G-protein coupled receptor rhodopsin. J. Am. Chem. Soc. 129, 6970-6971 (2007).
%Vashisth, H., Abrams, C.F. Ligand escape pathways and (un)binding free energy calculations for the hexameric insulin-phenol complex. Biophys. J. 95, 4193-4204 (2008).
%Perakyla, M. Ligand unbinding pathways from the vitamin D receptor studied by molecular dynamics simulations. 38, 185-198 (2009).doi:10.1007/s00249-008-0369-x 
%Klvana, M. et al. Pathways and Mechanisms for Product Release in the Engineered Haloalkane Dehalogenases Explored Using Classical and Random Acceleration Molecular Dynamics Simulations 
J. Mol. Biol. 392, 1339-1356 (2009).
%Pavlova, M. et al. Redesigning dehalogenase access tunnels as a strategy for degrading an anthropogenic substrate Nature Chem. Biol. 5, 727-733 (2009).
%Wang, T., Duan, Y. Ligand entry and exit pathways in the beta2-adrenergic receptor. J. Mol. Biol. 392, 1102-1115 (2009).



