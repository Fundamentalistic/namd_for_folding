%%%%%%%%%%%%%%%%%%%%%%%%%%%%%%%%%%%%%%%%%%%%%%%%%%%%%%%%%%%%%%%%%%%%%%%%%%%%
%                                                                          %
%              (C) Copyright 1995 The Board of Trustees of the             %
%                          University of Illinois                          %
%                           All Rights Reserved                            %
%								 	   %
%%%%%%%%%%%%%%%%%%%%%%%%%%%%%%%%%%%%%%%%%%%%%%%%%%%%%%%%%%%%%%%%%%%%%%%%%%%%

%%%%%%%%%%%%%%%%%%%%%%%%%%%%%%%%%%%%%%%%%%%%%%%%%%%%%%%%%%%%%%%%%%%%%%%%%%%%
% RCS INFORMATION:
%
%       $RCSfile: ug_psfgen.tex,v $
%       $Author: justin $        $Locker:  $                $State: Exp $
%       $Revision: 1.4 $      $Date: 2001/03/22 17:42:31 $
%
%%%%%%%%%%%%%%%%%%%%%%%%%%%%%%%%%%%%%%%%%%%%%%%%%%%%%%%%%%%%%%%%%%%%%%%%%%%%
% DESCRIPTION:
%	This section is part of the NAMD Users Guide and describes the
%	use of psfgen to create PSF input files.
%
%%%%%%%%%%%%%%%%%%%%%%%%%%%%%%%%%%%%%%%%%%%%%%%%%%%%%%%%%%%%%%%%%%%%%%%%%%%%

\section{Creating PSF Structure Files}
\label{section:psfgen}

The \verb#psfgen# structure building tool consists of a portable library
of structure and file manipulation routines with a Tcl interface.
Current capabilities include
\begin{itemize}
\item reading CHARMM topology files
\item reading psf files in X-PLOR/NAMD format
\item extracting sequence data from single segment PDB files
\item generating a full molecular structure from sequence data
\item applying patches to modify or link different segments
\item writing NAMD and VMD compatible PSF structure files
\item extracting coordinate data from single segment PDB files
\item constructing (guessing) missing atomic coordinates
\item writing NAMD and VMD compatible PDB coordinate files
\end{itemize}

We are currently refining the interface of \verb#psfgen# and adding
features to create a complete molecular building solution.  In the
meantime, documentation is provided by the following commented example.
We welcome your feedback on this new tool.

\subsection{BPTI Example}

To actually run this demo requires
\begin{itemize}
\item the program \verb#psfgen# from any \NAMD\ distribution,
\item the CHARMM topology and parameter files \verb#top_all22_prot.inp# and
\verb#par_all22_prot.inp# from 
{\tt https://rxsecure.umaryland.edu/research/amackere/research.html}, and
\item the BPTI PDB file \verb#6PTI.pdb# available from the Protein Data Bank at
{\tt http://www.pdb.org/} by searching for \verb#6PTI# and downloading
the complete structure file in PDB format.
\end{itemize}

In this demo, we create the files \verb#bpti.psf# and \verb#bpti.pdb# in the output directory
which can then be used for a simple NAMD simulation.

Create the working directory.  Nothing outside of the directory \verb#output# is modified.

\begin{verbatim}
mkdir output
\end{verbatim}

\paragraph*{Splitting input PDB file into segments.}

6PTI.pdb is the original file from the Protein Data Bank.  It contains
a single chain of protein and some PO4 and H2O HETATM records.  Since
each segment must have a separate input file, we remove all non-protein
atom records using grep.  If there were multiple chains we would have
to split the file by hand.
\begin{verbatim}
grep -v '^HETATM' 6PTI.pdb > output/6PTI_protein.pdb
\end{verbatim}

Create a second file containing only waters.
\begin{verbatim}
grep 'HOH' 6PTI.pdb > output/6PTI_water.pdb
\end{verbatim}


Run the psfgen program, taking everything until "ENDMOL" as input.
You may run psfgen interactively as well.  Since psfgen is built on
a Tcl interpreter, you may use loops, variables, etc., but you must
use \verb#$$# for variables when inside a shell script.  If you
want, run psfgen and enter the following commands manually.

\begin{verbatim}
psfgen << ENDMOL
\end{verbatim}

\paragraph*{Reading topology file.}

Read in the topology definitions for the residues we will create.
This must match the parameter file used for the simulation as well.
Multiple topology files may be read in since psfgen and NAMD use atom
type names rather than numbers in psf files.

\begin{verbatim}
topology toppar/top_all22_prot.inp
\end{verbatim}

\paragraph*{Building segment BPTI.}


Actually build a segment, calling it BPTI and reading the sequence
of residues from the stripped pdb file created above.  In addition to
the pdb command, we could specify residues explicitly.  Both angles
and dihedrals are generated automatically unless "auto none" is added
(which is required to build residues of water).  The commands "first"
and "last" may be used to change the default patches for the ends of
the chain.  The structure is built when the closing \} is encountered,
and some errors regarding the first and last residue are normal.

\begin{verbatim}
segment BPTI {
 pdb output/6PTI_protein.pdb
}
\end{verbatim}

\paragraph*{Adding patches.}


Some patch residues (those not used to begin or end a chain) are
applied after the segment is built.  These contain all angle and
dihedral terms explicitly since they were already generated.  In this
case we apply the patch for a disulfide link three separate times.

\begin{verbatim}
patch DISU BPTI:5 BPTI:55
patch DISU BPTI:14 BPTI:38
patch DISU BPTI:30 BPTI:51
\end{verbatim}

\paragraph*{Writing psf structure file.}


Now that all of the atoms and bonds have been created, we can write
out the psf structure file for the system.

\begin{verbatim}
writepsf output/bpti.psf
\end{verbatim}

\paragraph*{Reading coordinates from pdb file.}


The same file used to generate the sequence is now read to extract
coordinates.  In the residue ILE, the atom CD is called CD1 in the
pdb file, so we use "alias atom" to define the correct name.  Segment
names in the pdb file are ignored so we specify that the coordinates
should be applied to the segment BPTI.

\begin{verbatim}
alias atom ILE CD1 CD
coordpdb output/6PTI_protein.pdb BPTI
\end{verbatim}

\paragraph*{Guessing missing coordinates.}


The tolopogy file contains default internal coordinates which can be
used to guess the locations of many atoms, hydrogens in particular.
In the output pdb file, the occupancy field of guessed atoms will be
set to 1, atoms which are known are set to 0, and atoms which could
not be guessed are set to -1.  Some atoms are "poorly guessed" if
needed bond lengths and angles were missing from the topology file.

\begin{verbatim}
guesscoord
\end{verbatim}

\paragraph*{Writing pdb coordinate file.}


This ends the matching coordinate pdb file.  The psf and pdb files
are a matched set with identical atom ordering as needed by NAMD.

\begin{verbatim}
writepdb output/bpti.pdb
\end{verbatim}

\paragraph*{Adding a segment of water.}


Build a segment for the crystal waters.  The residue type for water
depends on the model, so here we alias HOH to TIP3.  Because CHARMM
uses an additional H-H bond we must disable generation of angles and
dihedrals for segments containing water.  Then read the pdb file.

\begin{verbatim}
alias residue HOH TIP3
segment SOLV {
 auto none
 pdb output/6PTI_water.pdb
}
\end{verbatim}

\paragraph*{Reading water coordinates.}


Alias the atom type for water oxygen as well and read coordinates from
the file to the segment SOLV.  Hydrogen doesn't show up in crystal
structures so it is missing from this pdb file.

\begin{verbatim}
alias atom HOH O OH2
coordpdb output/6PTI_water.pdb SOLV
\end{verbatim}

\paragraph*{Trying to guess water coordinates is not possible.}


Try to guess coordinates again, but hydrogens are not guessed for water
because all three atoms are needed to orient the molecules.

\begin{verbatim}
guesscoord
\end{verbatim}


Don't write out the structure with water since coordinates are missing.

\begin{verbatim}
ENDMOL
\end{verbatim}

\paragraph*{Using generated files in NAMD.}

The files bpti.pdb and bpti.psf can now be used with \NAMD, but the
initial coordinates require minimization first.
The following is an example \NAMD\ configuration file for the BPTI example.

\newpage
\begin{verbatim}
# NAMD configuration file for BPTI

# molecular system
structure	output/bpti.psf

# force field
paratypecharmm	on
parameters	toppar/par_all22_prot.inp
exclude		scaled1-4
1-4scaling	1.0

# approximations
switching	on
switchdist	8
cutoff		12
pairlistdist	13.5
margin		0
stepspercycle	20

#integrator
timestep 1.0

#output
outputenergies	10
outputtiming	100
binaryoutput	no

# molecular system
coordinates	output/bpti.pdb

#output
outputname	output/bpti
dcdfreq		1000

#protocol
temperature	0
reassignFreq	1000
reassignTemp	25
reassignIncr	25
reassignHold	300

#script

minimize 1000

run 20000
\end{verbatim}

%\subsection{Summary of psfgen commands}
%\begin{itemize}
%\item
%{\bf readpsf <filename>} : Reads the psf file given by filename and adds it to the
%molecular structure.  It is an error if any segments in the psf file have
%already been created.
%\end{itemize}

 
