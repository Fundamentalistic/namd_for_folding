
\subsection{Free Energy of Conformational Change Calculations}
\label{section:fenergy}

\subsubsection{User-Supplied Conformational Restraints}

These restraints extend the scope of the available restraints beyond that
provided by the harmonic position restraints. Each restraint is imposed with
a potential energy term, whose form depends on the type of the
restraint.\medskip

\paragraph*{Fixed Restraints}

{\em Position restraint (1 atom):} force constant $K_{f}$, and reference
position $\overrightarrow{r_{ref}}$

$\qquad \qquad \qquad \qquad E=\left( K_{f}/2\right) \left( \left| 
\overrightarrow{r_{i}}-\overrightarrow{r_{ref}}\right| \right) ^{2}$

{\em Stretch restraint (2 atoms):} force constant $K_{f}$, and reference
distance $d_{ref}$

$\qquad \qquad \qquad \qquad E=\left( K_{f}/2\right) \left(
d_{i}-d_{ref}\right) ^{2}$

{\em Bend restraint (3 atoms):} force constant $K_{f}$, and reference angle $%
\theta _{ref}$

$\qquad \qquad \qquad \qquad E=\left( K_{f}/2\right) \left( \theta
_{i}-\theta _{ref}\right) ^{2}$

{\em Torsion restraint (4 atoms):} energy barrier $E_{0}$, and reference
angle $\chi _{ref}$

$\qquad \qquad \qquad \qquad E=\left( E_{0}/2\right) \left\{ 1-\cos \left(
\chi _{i}-\chi _{ref}\right) \right\} $

\paragraph*{Forcing restraints}

{\em Position restraint (1 atom):} force constant $K_{f}$, and two reference
positions $\overrightarrow{r_{0}}$ and $\overrightarrow{r_{1}}$

$\qquad \qquad \qquad \qquad E=\left( K_{f}/2\right) \left( \left| 
\overrightarrow{r_{i}}-\overrightarrow{r_{ref}}\right| \right) ^{2}$

$\qquad \qquad \qquad \qquad \overrightarrow{r_{ref}}$ $=\lambda 
\overrightarrow{r_{1}}+\left( 1-\lambda \right) $ $\overrightarrow{r_{0}}$

{\em Stretch restraint (2 atoms):} force constant $K_{f}$, and two reference
distances $d_{0}$ and $d_{1}$

$\qquad \qquad \qquad \qquad E=\left( K_{f}/2\right) \left(
d_{i}-d_{ref}\right) ^{2}$

$\qquad \qquad \qquad \qquad d_{ref}=d_{1}+\left( 1-\lambda \right) d_{0}$

{\em Bend restraint (3 atoms):} force constant $K_{f}$, and two reference
angles $\theta _{0}$ and $\theta _{1}$

$\qquad \qquad \qquad \qquad E=\left( K_{f}/2\right) \left( \theta
_{i}-\theta _{ref}\right) ^{2}$

$\qquad \qquad \qquad \qquad \theta _{ref}=\lambda \theta _{1}+\left(
1-\lambda \right) \theta _{0}$

{\em Torsion restraint (4 atoms):} energy barrier E$_{0}$, and two reference
angles $\chi _{0}$ and $\chi _{1}$

$\qquad \qquad \qquad \qquad E=\left( E_{0}/2\right) \left\{ 1-\cos \left(
\chi _{i}-\chi _{ref}\right) \right\} $

$\qquad \qquad \qquad \qquad \chi _{ref}=\lambda \chi _{1}+\left( 1-\lambda
\right) \chi _{0}$

The forcing restraints depend on the coupling parameter, $\lambda $,
specified in a conformational forcing calculation. For example, the
restraint distance, $d_{ref}$, depends on $\lambda $, and as $\lambda $
changes two atoms or centers-of-mass are forced closer together or further
apart. In this case $K_{f}$ = $K_{f,0}$, the value supplied at input.

Alternatively, the value of $K_{f}$ may depend upon the coupling parameter $%
\lambda $ according to:

$K_{f}$ = $K_{f,0}$\pagebreak $\lambda $

\paragraph*{Bounds}

\begin{tabular}{ll}
{\em Position bound (1 atom):} & Force constant $K_{f}$, reference position $%
\overrightarrow{r_{ref}}$, \\ 
& and upper or lower reference distance, $d_{ref}$%
\end{tabular}

\qquad Upper bound:

$\qquad \qquad \qquad \qquad E=\left( K_{f}/2\right) \left(
d_{i}-d_{ref}\right) ^{2}$ for $d_{i}>d_{ref}$, else $E=0$.

\qquad Lower bound:

$\qquad \qquad \qquad \qquad E=\left( K_{f}/2\right) \left(
d_{i}-d_{ref}\right) ^{2}$ for $d_{i}$ $<$ $d_{ref}$, else $E=0$.\smallskip

$\qquad \qquad \qquad \qquad d_{i}^{2}=\left( \left| \overrightarrow{r_{i}}-%
\overrightarrow{r_{ref}}\right| \right) ^{2}\medskip \medskip $

\begin{tabular}{ll}
{\em Distance bound (2 atoms):} & Force constant $K_{f}$, \\ 
& and upper or lower reference distance, $d_{ref}$%
\end{tabular}

\qquad Upper bound:

$\qquad \qquad \qquad \qquad E=\left( K_{f}/2\right) \left(
d_{ij}-d_{ref}\right) ^{2}$ for $d_{ij}>d_{ref}$, else $E=0$.

\qquad Lower bound:

$\qquad \qquad \qquad \qquad E=\left( K_{f}/2\right) \left(
d_{ij}-d_{ref}\right) ^{2}$ for $d_{ij}<d_{ref}$, else $E=0$.\medskip
\medskip 

\begin{tabular}{ll}
{\em Angle bound (3 atoms):} & Force constant $K_{f}$, \\ 
& and upper or lower reference angle, $\theta _{ref}$%
\end{tabular}

\qquad Upper bound:

$\qquad \qquad \qquad \qquad E=\left( K_{f}/2\right) \left( \theta -\theta
_{ref}\right) ^{2}$ for $\theta >\theta _{ref},$ else $E=0$.

\qquad Lower bound:

$\qquad \qquad \qquad \qquad E=\left( K_{f}/2\right) \left( \theta -\theta
_{ref}\right) ^{2}$ for $\theta <\theta _{ref},$ else $E=0.\medskip \medskip
\medskip $

\begin{tabular}{ll}
{\em Torsion bound (4 atoms):} & An upper and lower bound must be provided
together. \\ 
& Energy gap $E_{0}$, lower AND upper reference angles, $\chi _{1}$ and $%
\chi _{2}$, \\ 
& and angle~interval, $\Delta \chi .$%
\end{tabular}

$\qquad \qquad 
\begin{tabular}{llll}
$\chi _{1}$ & $<\chi $ & $<\chi _{2}$ : & $E=0$ \\ 
$\left( \chi _{1}-\Delta \chi \right) $ & $<\chi $ & $<\chi _{1}$ : & $%
E=\left( G/2\right) \left\{ 1-\cos \left( \chi -\chi _{1}\right) \right\} $
\\ 
$\chi _{2}$ & $<\chi $ & $\left( \chi _{2}+\Delta \chi \right) $: & $%
E=\left( G/2\right) \left\{ 1-\cos \left( \chi -\chi _{2}\right) \right\} $
\\ 
$\left( \chi _{2}+\Delta \chi \right) ~$ & $<\chi $ & $\left( \chi
_{1}-\Delta \chi +2\pi \right) $ : & $E=G$%
\end{tabular}
$

$\qquad \qquad G=E_{0}/\left\{ 1-\cos \left( \Delta \chi \right) \right\}
\bigskip $

Bounds may be used in pairs, to set a lower and upper bound. Torsional
bounds always are defined in pairs.\pagebreak

\subsubsection{Free Energy Calculations}

\paragraph*{Conformational forcing / Potential of mean force}

In conformational forcing calculations, structural parameters such as atomic
positions, inter-atomic distances, and dihedral angles are forced to change
by application of changing restraint potentials. For example, the distance
between two atoms can be restrained by a potential to a mean distance that
is varied during the calculation. The free energy change (or potential of
mean force, pmf) for the process can be estimated during the simulation.

The potential is made to depend on a coupling parameter, $\lambda $, whose
value changes during the simulation. In potential of mean force
calculations, the reference value of the restraint potential depends on $%
\lambda $. Alternately, the force constant for the restraint potential may
change in proportion to the coupling parameter. Such a calculation gives the
value of a restraint free energy, i.e., the free energy change of the
syste\bigskip m due to imposition of the restraint potential.

\paragraph*{Methods for computing the free energy}

With conformational forcing (or with molecular transformation calculations)
one obtains a free energy difference for a process that is forced on the
system by changing the potential energy function that determines the
dynamics of the system. One always makes the changing potential depend on a
coupling parameter, $\lambda $. By convention, $\lambda $ can have values
only in the range from $0$ to $1$, and a value of $\lambda =0$ corresponds
to one defined state and a value of $\lambda =1$ corresponds to the other
defined state. Intermediate values of $\lambda $ correspond to intermediate
states; in the case of conformational forcing calculations these
intermediate states are physically realizable, but in the case of molecular
transformation calculations they are not.

The value of $\lambda $ is changed during the simulation. In the first
method provided here, the change in $\lambda $ is stepwise, while in the
second method it is virtually continuous.\medskip

\subparagraph*{Multi-configurational thermodynamic integration (MCTI).}

In MCTI one accumulates $\,\left\langle \partial U/\partial \lambda
\right\rangle $ at several values of $\lambda $, and from these averages
estimates the integral

\qquad \qquad \qquad \qquad $-\Delta A=\int \,\left\langle \partial U/%
\partial \lambda \right\rangle d\lambda $

With this method, the precision of each $\left\langle \partial U/\partial %
\lambda \right\rangle $ can be estimated from the fluctuations of the time
series of $\partial U/\partial \lambda $.\bigskip 

\subparagraph*{Slow growth.}

In slow growth, $\lambda $ is incremented by $\delta \lambda =\pm 1/N_{step}$
after each dynamics integration time-step, and the pmf is estimated as

\qquad \qquad \qquad \qquad $-\Delta A=\Sigma $ $\left( \partial U/\partial
\lambda \right) $ $\delta \lambda $

Typically, slow growth is done in cycles of: equilibration at $\lambda =0$,
change to $\lambda =1$, equilibration at $\lambda =1$, change to $\lambda =0$%
. It is usual to estimate the precision of slow growth simulations from the
results of successive cycles.\pagebreak 

\subsubsection{Options for Conformational Restraints}

\paragraph*{User-supplied restraint and bounds specifications}

\qquad \qquad urestraint $\{$

\qquad \qquad \quad n * (restraint or bound specification)\qquad \qquad //
see below

\qquad \qquad $\}\medskip $

\paragraph*{Restraint Specifications (not coupled to pmf calculations)}

\qquad \qquad 
\begin{tabular}{llll}
posi & ATOM & kf = KF & ref = (X Y Z) \\ 
dist & 2 x ATOM & kf = KF & ref = D \\ 
angle & 3 x ATOM & kf = KF & ref = A \\ 
dihe & 4 x ATOM & barr = B & ref = A
\end{tabular}
\bigskip 

\paragraph*{Bound Specifications (not coupled to pmf calculations)}

\qquad \qquad 
\begin{tabular}{llll}
posi bound & ATOM & kf = KF & [low = (X Y Z D) or hi = (X Y Z D)] \\ 
dist bound & 2 x ATOM & kf = KF & [low = D or hi = D] \\ 
angle bound & 3 x ATOM & kf = KF & [low = A or hi = A] \\ 
dihe bound & 4 x ATOM & gap = E & low = A0\quad hi = A1\quad delta = A2
\end{tabular}
\bigskip 

\paragraph*{Forcing Restraint Specifications (coupled to pmf calculations)}

\qquad \qquad 
\begin{tabular}{llll}
posi pmf & ATOM & kf=KF & low = (X0 Y0 Z0)\quad hi = (X1 Y1 Z1) \\ 
dist pmf & 2 x ATOM & kf=KF & low = D0\quad hi = D1 \\ 
angle pmf & 3 x ATOM & kf=KF & low = A0\quad hi = A1 \\ 
dihe pmf & 4 x ATOM & barr=B & low = A0\quad hi = A1
\end{tabular}
\bigskip 

\paragraph*{Units}

\qquad \qquad 
\begin{tabular}{|c|c|}
\hline
Input item & Units \\ \hline
E, B & kcal/mol \\ 
X, Y, Z, D & %TCIMACRO{\UNICODE{0xc5}{}}
\\ 
A & degrees \\ 
$K_{f}$ & kcal/(mol %TCIMACRO{\UNICODE{0xc5}{}}
$^{2}$) or kcal/(mol rad$^{2}$) \\ \hline
\end{tabular}

\pagebreak

\subsubsection{Options for ATOM Specification}

The designation ATOM, above, stands for one of the following forms:\medskip

\paragraph*{A single atom}

(segname, resnum, atomname)

{\em Example:} (insulin, 10, ca)\medskip 

\paragraph*{All atoms of a single residue}

(segname, resnum)

{\em Example:} (insulin, 10)\medskip 

\paragraph*{A list of atoms}

group $\{$ (segname, resnum, atomname), (segname, resnum, atomname), $\ldots 
$ $\}$

{\em Example:} group $\{$ (insulin, 10, ca), (insulin, 10, cb), (insulin,
11, cg) $\}\medskip $

\paragraph*{All atoms in a list of residues}

group $\{$ (segname, resnum), (segname, resnum), $\ldots $ $\}$

{\em Example:} group $\{$ (insulin, 10), (insulin, 12), (insulin, 14) $%
\}\medskip $

\paragraph*{All atoms in a range of residues}

group $\{$ (segname, resnum) to (segname, resnum) $\}$

{\em Example:} group $\{$ (insulin, 10) to (insulin, 12) $\}\medskip $

\paragraph*{One or more atomnames in a list of residues}

\begin{tabular}{l}
group $\{$ atomname: (segname, resnum), (segname, resnum), $\ldots $ $\}$ \\ 
group $\{$ (atomname, atomname, $\ldots $ ): (segname, resnum), (segname,
resnum), $\ldots $ $\}$%
\end{tabular}

\begin{tabular}{ll}
{\em Examples:} & group $\{$ ca: (insulin, 10), (insulin, 12), (insulin, 14) 
$\}$ \\ 
& group $\{$ (ca, cb, cg): (insulin, 10), (insulin, 12), (insulin, 14) $\}$
\\ 
& group $\{$ (ca, cb): (insulin, 10), (insulin, 12) cg: (insulin, 11),
(insulin, 12) $\}\smallskip $%
\end{tabular}
\medskip 

{\em Note: }Within a group, atomname is in effect until a new atomname is
used, or the keyword all is used. atomname will not carry over from group to
group. This note applies to the paragraph below.\medskip 

\paragraph*{One or more atomnames in a range of residues}

\begin{tabular}{l}
group $\{$ atomname: (segname, resnum) to (segname, resnum) $\}$ \\ 
group $\{$ (atomname, atomname, $\ldots $ ): (segname, resnum) to (segname,
resnum) $\}$%
\end{tabular}

\begin{tabular}{ll}
{\em Examples:} & group $\{$ ca: (insulin, 10) to (insulin, 14) $\}$ \\ 
& group $\{$ (ca, cb, cg): (insulin, 10) to (insulin, 12) $\}$ \\ 
& group $\{$ (ca, cb): (insulin, 10) to (insulin, 12) all: (insulin, 13) $\}$
\pagebreak 
\end{tabular}
\pagebreak 

\subsubsection{Options for Potential of Mean Force Calculation}

The pmf and mcti blocks, below, are used to simultaneously control all
forcing restraints specified in urestraint above. These blocks are performed
consecutively, in the order they appear in the config file. The pmf block is
used to either a) smoothly vary $\lambda $ from 0 $\rightarrow $1 or 1 $%
\rightarrow $0, or b) set lambda. The mcti block is used to vary $\lambda $
from 0 $\rightarrow $1 or 1 $\rightarrow $0 in steps, so that $\lambda $ is
fixed while $dU/d\lambda $ is accumulated.\medskip

\paragraph*{Lamba control for slow growth}

pmf $\{$

~~task = [up, down, stop, grow, fade, or nogrow]

~~time = T [fs, ps, or ns] (default = ps)

~~lambda = Y (value of $\lambda $; only needed for stop and nogrow)

~~lambdat = Z (value of $\lambda _{t}$; only needed for grow, fade, and
nogrow) (default = 0)

~~print = P [fs, ps, or ns] or noprint (default = ps)

$\}\medskip $

\begin{tabular}{ll}
up, down, stop: & $\lambda $ is applied to the reference values. \\ 
grow, fade, nogrow: & $\,\lambda $ is applied to $K_{f}$. A fixed value, $%
\lambda _{t}$, is used to determine the ref. values. \\ 
up, grow: & $\lambda $ changes from 0 $\rightarrow $1. (no value of $
\,\lambda $ is required) \\ 
down, fade: & $\lambda $ changes from 1 $\rightarrow $0. (no value of $%
\,\lambda $ is required) \\ 
stop, nogrow: & dU/d$\lambda $ is accumulated (for single point
MCTI)\medskip \smallskip
\end{tabular}
\bigskip

\paragraph*{Lambda control for automated MCTI}

mcti $\{$

~~task = [stepup, stepdown, stepgrow, or stepfade]

~~equiltime = T1 [fs, ps, or ns] (default = ps)

~~accumtime = T2 [fs, ps, or ns] (default = ps)

~~numsteps = N

~~lambdat = Z (value of $\lambda _{t}$; only needed for stepgrow, and
stepfade) (default = 0)

~~print = P [fs, ps, or ns] or noprint (default = ps)

$\}\medskip $

\begin{tabular}{ll}
stepup, stepdown: & $\lambda $ is applied to the reference values. \\ 
stepgrow, stepfade: & $\lambda $ is applied to $K_{f}$. A fixed value, $%
\lambda _{t}$, is used to determine the ref. values. \\ 
stepup, stepgrow: & $\lambda $ changes from 0 $\rightarrow $1. (no value of $%
\lambda $ is required) \\ 
stepdown, stepfade: & $\lambda $ changes from 1 $\rightarrow $0. (no value
of $\lambda $ is required)\medskip
\end{tabular}

For each task, $\lambda $ changes in steps of (1.0/N) from 0 $\rightarrow $1
or 1 $\rightarrow $0. At each step, no data is accumulated for the initial
period T1, then dU/d$\lambda $ is accumulated for T2. Therefore, the total
duration of an mcti block is (T1+T2) x N.

\subsubsection{Examples}

\paragraph*{Fixed restraints}

\begin{tabular}{l}
{\footnotesize // 1. restrain the position of the ca atom of residue 0.} \\ 
{\footnotesize // 2. restrain the distance between the ca's of residues 0
and 10 to 5.2\AA } \\ 
{\footnotesize // 3. restrain the angle between the ca's of residues 0-10-20
to 90}$^{o}${\footnotesize \ .} \\ 
{\footnotesize // 4. restrain the dihedral angle between the ca's of
residues 0-10-20-30 to 180}$^{o}${\footnotesize \ .} \\ 
{\footnotesize // 5. restrain the angle between the centers-of-mass of
residues 0-10-20 to 90}$^{o}${\footnotesize \ .}
\end{tabular}

urestraint $\{$

~~posi (insulin, 0, ca) kf=20 ref=(10, 11, 11)

~~dist (insulin, 0, ca) (insulin, 10, ca) kf=20 ref=5.2

~~angle (insulin, 0, ca) (insulin, 10, ca) (insulin, 20, ca) kf=20 ref=90

~~dihe (insulin, 0, ca) (insulin, 10, ca) (insulin, 20, ca) (insulin, 30,
ca) barr=20 ref=180

~~angle (insulin, 0) (insulin, 10) (insulin, 20) kf=20 ref=90

$\}\bigskip $

\begin{tabular}{ll}
{\footnotesize // 1. } & {\footnotesize restrain the center of mass of three
atoms of residue 0.} \\ 
{\footnotesize // 2.} & {\footnotesize restrain the distance between (the
COM of 3 atoms of residue 0) to  (the COM of 3 atoms of residue 10).} \\ 
{\footnotesize // 3.} & {\footnotesize \ restrain the dihedral angle of
(10,11,12)-(15,16,17,18)-(20,22)-(30,31,32,34,35) to 90}$^{o}$ \\ 
{\footnotesize //} & {\footnotesize ( (ca of 10 to 12), (ca, cb, cg of 15 to
18), (all atoms of 20 and 22), (ca of 30, 31, 32, 34, all atoms of 35) ).}
\end{tabular}

urestraint $\{$

~~posi group $\{$(insulin, 0, ca), (insulin, 0, cb), (insulin, 0, cg)$\}$
kf=20 ref=(10, 11, 11)

~~
\begin{tabular}{ll}
dist & group $\{$(insulin, 0, ca), (insulin, 0, cb), (insulin, 0, cg)$\}$ \\ 
& group $\{$(insulin, 10, ca), (insulin, 10, cb), (insulin, 10, cg)$\}$
kf=20 ref=6.2
\end{tabular}

~~
\begin{tabular}{ll}
dihe & group $\{$ca: (insulin, 10) to (insulin, 12)$\}$ \\ 
& group $\{$(ca, cb, cg): (insulin, 15) to (insulin, 18)$\}$ \\ 
& group $\{$(insulin, 20), (insulin, 22)$\}$ \\ 
& group $\{$ca: (insulin, 30) to (insulin, 32), (insulin, 34), all:
(insulin, 35)$\}$ barr=20 ref=90
\end{tabular}

$\}$

\paragraph*{Bound specifications}

\begin{tabular}{ll}
{\footnotesize // 1. } & {\footnotesize impose an upper bound if an atom's
position strays too far from a reference position.} \\ 
{\footnotesize // } & {\footnotesize (add an energy term if the atom is more
than 10\AA\ {}from (2.0, 2.0, 2.0) ).} \\ 
{\footnotesize // 2\&3.} & {\footnotesize \ impose lower and upper bounds on
the distance between the ca's of residues 5 and 15.} \\ 
{\footnotesize //} & {\footnotesize (if the separation is less than 5.0\AA\ %
{}or greater than 12.0\AA\ {}add an energy term).} \\ 
{\footnotesize // 4.} & {\footnotesize \ impose a lower bound on the angle
between the centers-of-mass of residues 3-6-9.} \\ 
{\footnotesize //} & {\footnotesize (if the angle goes lower than 90}$^{o}$%
{\footnotesize \ apply a restraining potential).}
\end{tabular}

urestraint $\{$

~~posi bound (insulin, 3, cb) kf=20 hi = (2.0, 2.0, 2.0, 10.0)

~~dist bound (insulin, 5, ca) (insulin, 15, ca) kf=20 low = 5.0

~~dist bound (insulin, 5, ca) (insulin, 15, ca) kf=20 hi = 12.0

~~angle bound (insulin, 3) (insulin, 6) (insulin, 9) kf=20 low=90.0

$\}\bigskip $

\begin{tabular}{l}
{\footnotesize // torsional bounds are defined as pairs. this example
specifies upper and lower bounds on the} \\ 
{\footnotesize // dihedral angle, }$\chi ${\footnotesize {}, separating the
planes of the 1-2-3 residues and the 2-3-4 residues.}
\end{tabular}

%\begin{Body Text}
\begin{tabular}{llll}
{\footnotesize // The energy is 0 for:} & {\footnotesize -90}$^{o}$ & 
{\footnotesize < }$\chi $ & {\footnotesize <\ 120}$%
^{o}$ \\ 
{\footnotesize // The energy is 20 kcal/mol for:} & {\footnotesize 130}$^{o}$
& {\footnotesize <\ }$\chi $ & {\footnotesize <\
260}$^{o}$%
\end{tabular}

\begin{tabular}{l}
{\footnotesize // Energy rises from 0 }$\rightarrow ${\footnotesize \ 20
kcal/mol as }$\chi ${\footnotesize \ {}increases from 120}$^{o}\rightarrow $%
{\footnotesize \ 130}$^{o}${\footnotesize \ , and decreases from --90}$%
^{o}\rightarrow ${\footnotesize \ --100}$^{o}${\footnotesize .}
\end{tabular}
%\end{Body Text}

urestraint $\{$

~~dihe bound (insulin 1) (insulin 2) (insulin 3) (insulin 4) gap=20 low=-90
hi=120 delta=10

$\}$

\paragraph*{Forcing restraints}

\begin{tabular}{l}
{\footnotesize // a forcing restraint will be imposed on the distance
between the centers-of-mass of residues (10 to 15) and} \\ 
{\footnotesize // residues (30 to 35). low=20.0, hi=10.0, indicates that the
reference distance is 20.0%TCIMACRO{\UNICODE{0xc5}{}}
at }$\lambda ${\footnotesize =0, and 10.0%TCIMACRO{\UNICODE{0xc5}{}}
at }$\lambda ${\footnotesize =1.}
\end{tabular}

urestraint $\{$

~~
\begin{tabular}{ll}
dist pmf & group $\{$ (insulin, 10) to (insulin, 15) $\}$ \\ 
& \hspace{0pt}group $\{$ (insulin, 30) to (insulin, 35) $\}$ kf=20,
low=20.0, hi=10.0
\end{tabular}

$\}\medskip $

\begin{tabular}{l}
{\footnotesize // 1. during the initial 10 ps, increase the strength of the
forcing restraint to full strength: 0 }$\rightarrow $ {\footnotesize 20
kcal/(mol %TCIMACRO{\UNICODE{0xc5}{}}
}$^{2}${\footnotesize )} \\ 
{\footnotesize // 2. next, apply a force to slowly close the distance from
20 %TCIMACRO{\UNICODE{0xc5}{}}
to 10 %TCIMACRO{\UNICODE{0xc5}{}}
(}$\lambda ${\footnotesize \ changes from 0 }$\rightarrow ${\footnotesize \
1)} \\ 
{\footnotesize // 3. accumulate dU/d}$\lambda ${\footnotesize \ for another
10 ps. ( stays fixed at 1)} \\ 
{\footnotesize // 4. force the distance back to its initial value of 20 
%TCIMACRO{\UNICODE{0xc5}{}}
( changes from 1 }$\rightarrow $ {\footnotesize 0)}
\end{tabular}

pmf $\{$

~~task = grow

~~time = 10 ps

~~print = 1 ps

$\}$

pmf $\{$

~~task = up

~~time = 100 ps

$\}$

pmf $\{$

~~task = stop

~~time = 10 ps

$\}$

pmf $\{$

~~task = down

~~time = 100 ps

$\}\medskip $

\begin{tabular}{ll}
{\footnotesize // 1. } & {\footnotesize force the distance to close from 20 
%TCIMACRO{\UNICODE{0xc5}{}}
to 10 %TCIMACRO{\UNICODE{0xc5}{}}
in 5 steps. (}$\lambda ${\footnotesize \ changes from 0 }$\rightarrow $%
{\footnotesize \ 1: ~~0.2, 0.4, 0.6, 0.8, 1.0)} \\ 
{\footnotesize // } & {\footnotesize at each step equilibrate for 10 ps,
then collect dU/d}$\lambda ${\footnotesize \ for another 10 ps.} \\ 
{\footnotesize //} & {\footnotesize ref = 18, 16, 14, 12, 10 
%TCIMACRO{\UNICODE{0xc5}{}}
, duration = (10 + 10) x 5 = 100 ps.} \\ 
{\footnotesize // 2.} & {\footnotesize \ reverse the step above (}$\lambda $%
{\footnotesize \ changes from 1 }$\rightarrow $ {\footnotesize 0: ~~0.8,
0.6, 0.4, 0.2, 0.0)}
\end{tabular}

mcti $\{$

~~task = stepup

~~equiltime = 10 ps

~~accumtime = 10 ps

~~numsteps = 5

~~print = 1 ps

$\}$

mcti $\{$

~~task = stepdown

$\}$\pagebreak

\subsubsection{Appendix}

\paragraph*{Gradient for position restraint}

$E=\left( K_{f}/2\right) \left( \left| \overrightarrow{r_{i}}-%
\overrightarrow{r_{ref}}\right| \right) ^{2}$

$E=\left( K_{f}/2\right) \left\{ \left( x_{i}-x_{ref}\right) ^{2}+\left(
y_{i}-y_{ref}\right) ^{2}+\left( z_{i}-z_{ref}\right) ^{2}\right\} $

$\nabla (E)=K_{f}\left\{ \left( x_{i}-x_{ref}\right) \overrightarrow{i}%
+\left( y_{i}-y_{ref}\right) \overrightarrow{j}+\left( z_{i}-z_{ref}\right) 
\overrightarrow{k}\right\} $

\paragraph*{Gradient for stretch restraint}

$E=\left( K_{f}/2\right) \left( d_{i}-d_{ref}\right) ^{2}$

$d_{i}=\left\{ \left( x_{2}-x_{1}\right) ^{2}+\left( y_{2}-y_{1}\right)
^{2}+\left( z_{2}-z_{1}\right) ^{2}\right\} ^{1/2}$

$\nabla (E)=K_{f}\left( d_{i}-d_{ref}\right) \cdot \nabla (di)\medskip $

\subparagraph*{for atom 2 moving, and atom 1 fixed}

$\nabla (d_{i})=1/2\left\{ \left( x_{2}-x_{1}\right) ^{2}+\left(
y_{2}-y_{1}\right) ^{2}+\left( z_{2}-z_{1}\right) ^{2}\right\}
^{-1/2}\left\{ 2\left( x_{2}-x_{1}\right) +2\left( y_{2}-y_{1}\right)
+2\left( z_{2}-z_{1}\right) \right\} $

$\nabla (d_{i})=\left\{ \left( x_{2}-x_{1}\right) \overrightarrow{i}+\left(
y_{2}-y_{1}\right) \overrightarrow{j}+\left( z_{2}-z_{1}\right) 
\overrightarrow{k}\right\} /d_{i}$

$\nabla (E)=K_{f}\left\{ \left( d_{i}-d_{ref}\right) /d_{i}\right\} \left\{
\left( x_{2}-x_{1}\right) \overrightarrow{i}+\left( y_{2}-y_{1}\right) 
\overrightarrow{j}+\left( z_{2}-z_{1}\right) \overrightarrow{k}\right\} $

\paragraph*{Gradient for bend restraint}

$E=\left( K_{f}/2\right) \left( \theta _{i}-\theta _{ref}\right) ^{2}$

Atoms at positions A-B-C

distances: (A to B) = c; (A to C) = b; (B to C) = a;

$\theta _{i}=\cos ^{-1}(u)=\cos ^{-1}\left\{ \left( a^{2}+c^{2}-b^{2}\right)
/\left( 2ac\right) \right\} $

$\nabla (E)=K_{f}\left( \theta _{i}-\theta _{ref}\right) \cdot \nabla
(\theta _{i})$

$\nabla (\theta _{i})=\frac{-1}{\sqrt{1-u^{2}}}\cdot \nabla (u)\medskip $

\subparagraph*{for atom A moving, atoms B \& C fixed (distances b and c
change)}

$\nabla (u)=\left\{ -b/\left( ac\right) \right\} \cdot \nabla (b)+\left\{
-a/\left( 2c^{2}\right) +1/\left( 2a\right) +b^{2}/\left( 2ac^{2}\right)
\right\} \cdot \nabla (c)$

$\nabla (b)=\left\{ \left( x_{A}-x_{C}\right) \overrightarrow{i}+\left(
y_{A}-y_{C}\right) \overrightarrow{j}+\left( z_{A}-z_{C}\right) 
\overrightarrow{k}\right\} /b$

$\nabla (c)=\left\{ \left( x_{A}-x_{B}\right) \overrightarrow{i}+\left(
y_{A}-y_{B}\right) \overrightarrow{j}+\left( z_{A}-z_{B}\right) 
\overrightarrow{k}\right\} /c\medskip $

\subparagraph*{for atom B moving, atoms A \& C fixed (distances a and c
change)}

$\nabla (u)=\left\{ 1/(2c)+-c/(2a^{2})+b^{2}/(2a^{2}c)\right\} \cdot \nabla
(a)+\left\{ -a/\left( 2c^{2}\right) +1/(2a)+b^{2}/\left( 2ac^{2}\right)
\right\} \cdot \nabla (c)$

$\nabla (a)=\left\{ (x_{B}-x_{C})\overrightarrow{i}+(y_{B}-y_{C})%
\overrightarrow{j}+(z_{B}-z_{C})\overrightarrow{k}\right\} /a$

$\nabla (c)=\left\{ (x_{B}-x_{A})\overrightarrow{i}+(y_{B}-y_{A})%
\overrightarrow{j}+(z_{B}-z_{A})\overrightarrow{k}\right\} /c\medskip $

\subparagraph*{for atom C moving, atoms A \& B fixed (distances a and b
change)}

$\nabla (u)=\left\{ -b/\left( ac\right) \right\} \cdot \nabla (b)+\left\{
-c/\left( 2a^{2}\right) +1/(2c)+b^{2}/\left( 2ac^{2}\right) \right\} \cdot 
\nabla (a)$

$\nabla (b)=\left\{ (x_{C}-x_{A})\overrightarrow{i}+(y_{C}-y_{A})%
\overrightarrow{j}+(z_{C}-z_{A})\overrightarrow{k}\right\} /b$

$\nabla (a)=\left\{ (x_{C}-x_{B})\overrightarrow{i}+(y_{C}-y_{B})%
\overrightarrow{j}+(z_{C}-z_{B})\overrightarrow{k}\right\} /a $

\paragraph*{Gradient for dihedral angle restraint}

$E=(E_{0}/2)\left\{ 1-\cos \left( \chi _{i}-\chi _{ref}\right) \right\} $

Atoms at positions A-B-C-D

$\chi _{i}=\cos ^{-1}(u)=$ $\cos ^{-1}\left( \frac{\overrightarrow{(CD}%
\times \overrightarrow{CB})\bullet (\overrightarrow{BC}\times 
\overrightarrow{BA})}{\left| \overrightarrow{CD}\times \overrightarrow{CB}%
\right| \left| \overrightarrow{BC}\times \overrightarrow{BA}\right| }\right)
\qquad $AND

$\chi _{i}=\sin ^{-1}(v)=$ $\sin ^{-1}\left( \frac{\overrightarrow{(CD}%
\times \overrightarrow{CB})\times (\overrightarrow{BC}\times \overrightarrow{%
BA})}{\left| \overrightarrow{CD}\times \overrightarrow{CB}\right| \left| 
\overrightarrow{BC}\times \overrightarrow{BA}\right| }\bullet \frac{%
\overrightarrow{CB}}{\left| \overrightarrow{CB}\right| }\right) $

$\nabla (E)=(E_{0}/2)\left\{ \sin \left( \chi _{i}-\chi _{ref}\right)
\right\} \cdot \nabla (\chi _{i})$

$\nabla (\chi _{i})=\frac{-1}{\sqrt{1-u^{2}}}\cdot \nabla (u)\smallskip
\medskip $

\begin{tabular}{ll}
$\overrightarrow{CD}\times \overrightarrow{CB}$ $=$ & $%
((y_{D}-y_{C})(z_{B}-z_{C})-(z_{D}-z_{C})(y_{B}-y_{C}))\overrightarrow{i}+$
\\ 
& $((z_{D}-z_{C})(x_{B}-x_{C})-(x_{D}-x_{C})(z_{B}-z_{C}))\overrightarrow{j}+
$ \\ 
& $((x_{D}-x_{C})(y_{B}-y_{C})-(y_{D}-y_{C})(x_{B}-x_{C}))\overrightarrow{k}$
\\ 
\multicolumn{1}{c}{$=$} & $p_{1}\overrightarrow{i}+p_{2}\overrightarrow{j}%
+p_{3}\overrightarrow{k}$%
\end{tabular}
\medskip \medskip 

\begin{tabular}{ll}
$\overrightarrow{BC}\times \overrightarrow{BA}=$ & $%
((y_{C}-y_{B})(z_{A}-z_{B})-(z_{C}-z_{B})(y_{A}-y_{B}))\overrightarrow{i}+$
\\ 
& $((z_{C}-z_{B})(x_{A}-x_{B})-(x_{C}-x_{B})(z_{A}-z_{B}))\overrightarrow{j}+
$ \\ 
& $((x_{C}-x_{B})(y_{A}-y_{B})-(y_{C}-y_{B})(x_{A}-x_{B}))\overrightarrow{k}$
\\ 
\multicolumn{1}{c}{$=$} & $p_{4}\overrightarrow{i}+p_{5}\overrightarrow{j}
+p_{6}\overrightarrow{k}$%
\end{tabular}
\medskip \medskip 

$u=\frac{p_{1}p_{4}+p_{2}p_{5}+p_{3}p_{6}}{\sqrt{%
p_{1}^{2}+p_{2}^{2}+p_{3}^{2}}\sqrt{p_{4}^{2}+p_{5}^{2}+p_{6}^{2}}}\medskip
\medskip $

\begin{tabular}{ll}
$\nabla (u)=$ & $\frac{p_{1}\cdot \nabla (p_{4})+p_{2}\cdot \nabla
(p_{5})+p_{3}\cdot \nabla (p_{6})}{\sqrt{p_{1}^{2}+p_{2}^{2}+p_{3}^{2}}\sqrt{%
p_{4}^{2}+p_{5}^{2}+p_{6}^{2}}}$ $+$ \\ 
& $\frac{p_{1}p_{4}+p_{2}p_{5}+p_{3}p_{6}}{\sqrt{%
p_{1}^{2}+p_{2}^{2}+p_{3}^{2}}}$ $\left( -1/2\left(
p_{4}^{2}+p_{5}^{2}+p_{6}^{2}\right) ^{-3/2}\left( 2p_{4}\cdot \nabla
(p_{4})+2p_{5}\cdot \nabla (p_{5})+2p_{6}\cdot \nabla (p_{6})\right) \right) 
$ $+$ \\ 
& $\frac{p_{1}p_{4}+p_{2}p_{5}+p_{3}p_{6}}{\sqrt{%
p_{4}^{2}+p_{5}^{2}+p_{6}^{2}}}$ $\left( -1/2\left(
p_{1}^{2}+p_{2}^{2}+p_{3}^{2}\right) ^{-3/2}\left( 2p_{1}\cdot \nabla
(p_{1})+2p_{2}\cdot \nabla (p_{2})+2p_{3}\cdot \nabla (p_{3})\right) \right) 
$%
\end{tabular}
\pagebreak 

\subparagraph*{for atom A moving, atoms B, C, \& D fixed}

\begin{tabular}{lrrr}
$\nabla (p_{1})=$ & $(0.0)\overrightarrow{i}+$ & $(0.0)\overrightarrow{j}+$
& $(0.0)\overrightarrow{k}$ \\ 
$\nabla (p_{2})=$ & $(0.0)\overrightarrow{i}+$ & $(0.0)\overrightarrow{j}+$
& $(0.0)\overrightarrow{k}$ \\ 
$\nabla (p_{3})=$ & $(0.0)\overrightarrow{i}+$ & $(0.0)\overrightarrow{j}+$
& $(0.0)\overrightarrow{k}$ \\ 
$\nabla (p_{4})=$ & $(0.0)\overrightarrow{i}+$ & $(z_{B}-z_{C})%
\overrightarrow{j}+$ & $(y_{C}-y_{B})\overrightarrow{k}$ \\ 
$\nabla (p_{5})=$ & $(z_{C}-z_{B})\overrightarrow{i}+$ & $(0.0)%
\overrightarrow{j}+$ & $(x_{B}-x_{C})\overrightarrow{k}$ \\ 
$\nabla (p_{6})=$ & $(y_{B}-y_{C})\overrightarrow{i}+$ & $(x_{C}-x_{B})%
\overrightarrow{j}+$ & $(0.0)\nolinebreak \bigskip \overrightarrow{k}$%
\end{tabular}
\bigskip 

\subparagraph*{for atom B moving, atoms A, C, \& D fixed}

\begin{tabular}{lrrr}
$\nabla (p_{1})=$ & $(0.0)\overrightarrow{i}+$ & $(z_{C}-z_{D})%
\overrightarrow{j}+$ & $(y_{D}-y_{C})\overrightarrow{k}$ \\ 
$\nabla (p_{2})=$ & $(z_{D}-z_{C})\overrightarrow{i}+$ & $(0.0)%
\overrightarrow{j}+$ & $(x_{C}-x_{D})\overrightarrow{k}$ \\ 
$\nabla (p_{3})=$ & $(y_{C}-y_{D})\overrightarrow{i}+$ & $(x_{D}-x_{C})%
\overrightarrow{j}+$ & $(0.0)\overrightarrow{k}$ \\ 
$\nabla (p_{4})=$ & $(0.0)\overrightarrow{i}+$ & $(z_{C}-z_{A})%
\overrightarrow{j}+$ & $(y_{A}-y_{C})\overrightarrow{k}$ \\ 
$\nabla (p_{5})=$ & $(z_{A}-z_{C})\overrightarrow{i}+$ & $(0.0)%
\overrightarrow{j}+$ & $(x_{C}-x_{A})\overrightarrow{k}$ \\ 
$\nabla (p_{6})=$ & $(y_{C}-y_{A})\overrightarrow{i}+$ & $(x_{A}-x_{C})%
\overrightarrow{j}+$ & $(0.0)\overrightarrow{k}$%
\end{tabular}
\pagebreak 

\paragraph*{Gradient for forcing position restraint, $\partial U/\partial \lambda $}

$E=(K_{f}/2)\left( \left| \overrightarrow{r_{i}}-\overrightarrow{r_{ref}}%
\right| \right) ^{2}$

$r_{ref}=\lambda \overrightarrow{r_{1}}+\left( 1-\lambda \right) 
\overrightarrow{r_{0}}$

\begin{tabular}{lll}
$dE/d\lambda =$ & $K_{f}\times $ & $\left( \left( x_{i}-x_{ref}\right)
^{2}+\left( y_{i}-y_{ref}\right) ^{2}+\left( z_{i}-z_{ref}\right)
^{2}\right) ^{1/2}\times $ \\ 
&  & $1/2\left( \left( x_{i}-x_{ref}\right) ^{2}+\left( y_{i}-y_{ref}\right)
^{2}+\left( z_{i}-z_{ref}\right) ^{2}\right) ^{-1/2}\times $ \\ 
&  & $\left( 2\left( x_{i}-x_{ref}\right) \left( x_{0}-x_{1}\right) +2\left(
y_{i}-y_{ref}\right) \left( y_{0}-y_{1}\right) +2\left( z_{i}-z_{ref}\right)
\left( z_{0}-z_{1}\right) \right) $%
\end{tabular}

$dE/d\lambda =K_{f}\times \left( \left( x_{i}-x_{ref}\right) \left(
x_{0}-x_{1}\right) +\left( y_{i}-y_{ref}\right) \left( y_{0}-y_{1}\right)
+\left( z_{i}-z_{ref}\right) \left( z_{0}-z_{1}\right) \right) \bigskip $

\paragraph*{Gradient for forcing stretch restraint, $\partial U/\partial \lambda $}

$E=\left( K_{f}/2\right) \left( d_{i}-d_{ref}\right) ^{2}$

$d_{ref}=\lambda d_{1}+\left( 1-\lambda \right) d_{0}$

$dE/d\lambda =K_{f}~\times ~\left( d_{i}-d_{ref}\right) ~\times ~\left(
d_{0}-d_{1}\right) \bigskip $

\paragraph*{Gradient for forcing bend restraint, $\partial U/\partial \lambda $}

$E=\left( K_{f}/2\right) \left( \theta _{i}-\theta _{ref}\right) ^{2}$

$\theta _{ref}=\lambda \theta _{1}+\left( 1-\lambda \right) \theta _{0}$

$dE/d\lambda =K_{f}~\times ~\left( \theta _{i}-\theta _{ref}\right) ~\times %
~\left( \theta _{0}-\theta _{1}\right) \bigskip $

\paragraph*{Gradient for forcing dihedral restraint, $\partial U/\partial \lambda $}

$E=\left( E_{0}/2\right) \left( 1-\cos \left( \chi _{i}-\chi _{ref}\right)
\right) $

$\chi _{ref}=\lambda \chi _{1}+\left( 1-\lambda \right) \chi _{0}$

$dE/d\lambda =\left( E_{0}/2\right) ~\times ~\sin \left( \chi _{i}-\chi
_{ref}\right) ~\times ~\left( \chi _{0}-\chi _{1}\right) $

